\section{Skein Theory Generalities}
The goal of this section is to develop the tools needed to prove evaluability in a generalized class of trivalent categories.
Both of the categories studied in this paper are extensions of trivalent categories by a labeled, colored, directed, $\Z_n$-like strand, which we now begin to define. 

\begin{definition}
    A family
    \[
        \skein{/skein_figs/g}{0.15}_{i}
    \]
    of colored,directed strands, $i=1,\dots,n-1$ is a {\bf $\Z_n$-like family of strands} if they enjoy the following properties:

    \begin{equation*}\tag{Product}
        \skein{/skein_figs/g}{0.15}_{i} \skein{/skein_figs/g}{0.15}_{j} = \skein{/skein_figs/g}{0.15}_{i+j(\mod{n})}
    \end{equation*}

    \begin{equation*}\tag{Recouple}
        \skein{/skein_figs/recouple_LHS}{0.15} = \skein{/skein_figs/recouple_RHS}{0.15}
    \end{equation*}
    where a subscript of 0 is the empty strand (i.e., no strand at all) and a lack of subscript is understood to mean a subscript of 1.
\end{definition}


Recall the definition of {\it box spaces} from, e.g., \cite{tricats}.
\begin{definition}
    Let $\CC\cong \Kar\left( \left\langle \skein{/skein_figs/trivalent}{0.1} \right\rangle \right)$ be a trivalent category with $B(k,1) \leq B(k,0)$ for $k=1,\dots,5$. Call $\DD$ a {\bf $\Z_n$-like extension of $\CC$} if we have a containment $\CC\hookrightarrow \DD\cong \Kar\left(\left\langle \skein{/skein_figs/trivalent}{0.15}, \skein{/skein_figs/Pg}{0.15} \right\rangle \right)$, where $ \skein{/skein_figs/g}{0.15}_i$ is a $\Z_n$-like family of strands. 
\end{definition}


With this definition in mind we now define some useful notation for giving relations in such a category.
\begin{definition}
    Suppose a diagram $E$ has $m$ boundary points. We define $dec_i(E)$ to be the $i$-th external single clockwise decorations of $E$. For example,
    \[
        dec_1\left(\skein{/skein_figs/trivalent}{0.15}\right) = \skein{/skein_figs/rightDown}{0.1},
    \quad
    \text{and} 
    \quad
        \sum_{i=1}^{m} dec_i\left(\skein{/skein_figs/trivalent}{0.15}\right) = \skein{/skein_figs/rightDown}{0.1} +\skein{/skein_figs/bottomLeft}{0.1} +\skein{/skein_figs/leftUp}{0.1}
    \]
    We adopt the convention that $dec_0 (E) = E$.
\end{definition}



General methods for demonstrating evaluability of a skein theory involve identifying some measure of complexity for a linear combination of diagrams, and then showing the known relations allow one to strictly decrease this measure to zero. For the underlying trivalent category, the constraint on the box space dimensions allows us to decrement one measure of complexity: number of internal faces. With the new strand type, we have another measure: number of colored strands. We concatenate face count with colored strand count to obtain a lexicographic measure of complexity. The following result gives sufficient conditions to decrement this complexity.


\begin{proposition}\label{prop:eval-criteria}
    A $\Z_n$-like extension $\left\langle \skein{/skein_figs/trivalent}{0.15}, \skein{/skein_figs/Pg}{0.15} \right\rangle$ of a trivalent category $\left\langle \skein{/skein_figs/trivalent}{0.15} \right\rangle$ is evaluable if it enjoys the following relations:

    \begin{equation*}\tag{Change of Basis}
        \skein{/skein_figs/leftUp}{0.1} = \sum_{i=0}^{n-1} r_i \skein{/skein_figs/rightUp}{0.1}^i
    \end{equation*}

    \begin{equation*}\tag{Swap}
        \skein{/skein_figs/swap_LHS}{0.15} = \omega \skein{/skein_figs/swap_RHS}{0.15}
    \end{equation*}

    \begin{equation*}\tag{Slide}
        \skein{/skein_figs/slide_LHS}{0.1} = \sum_{i=0}^{n-1} s_i \skein{/skein_figs/slide_RHS2}{0.1}^i
    \end{equation*}

     \begin{equation*}\tag{decStick}
        \skein{/skein_figs/stick_w_Pg}{0.25} = a \skein{/skein_figs/stick}{0.1}
    \end{equation*}

    \begin{equation*}\tag{decBigon}
        \skein{/skein_figs/g_bigon1}{0.1}, \skein{/skein_figs/g_bigon2}{0.1} \in \C \skein{/skein_figs/stick}{0.15}
    \end{equation*}

     \begin{equation*}\tag{decTrigon}
        \skein{/skein_figs/trigon_singly}{0.1} = \sum_{i=0}^{n-1} t_i \skein{/skein_figs/rightUp}{0.1}^i
    \end{equation*}\label{eq:decTrigon}

    \begin{equation*}\tag{decTetragon}
         \skein{/skein_figs/square_singly}{0.12} = \sum_{i=0}^4 \sum_{j=0}^3 \alpha_{i,j} dec_i\left( r^j\left( \skein{/skein_figs/pent_1}{0.2} \right) \right)
    \end{equation*}

     \begin{equation*}\tag{decPentagon}
        \skein{/skein_figs/pentagon_singly}{0.17} = \sum_{i=0}^5 \sum_{j=0}^4 \beta_{i,j} dec_i\left( r^j\left( \skein{/skein_figs/pent_1}{0.2} \right) \right) + \sum_{i=0}^5 \sum_{j=0}^4 \gamma_{i,j} dec_i\left( r^j\left( \skein{/skein_figs/pent_2}{0.12} \right) \right)
    \end{equation*}
    
\end{proposition}


Under the conditions of \ref{prop:eval-criteria} we obtain the following lemma.

\begin{lemma}\label{lem:ext-dec}
    A decorated diagram may be expressed as a combination of externally decorated diagrams
\end{lemma}
\begin{proof}
    We prove the lemma for a decorated trigon, and leave the remaining cases to the reader. Consider a maximally-decorated trigon:
    \[
        \skein{/skein_figs/dec_3gon1}{0.2}
    \]
    with any labeling on the colored strands. We apply the relations (Swap) and (Slide) on the internal colored strands to obtain a combination of diagrams of the form
    \[
        \skein{/skein_figs/dec_3gon2}{0.2}
    \]
    Now apply (Change of Basis) to reduce to a combination of diagrams of the form
    \[
        \skein{/skein_figs/dec_3gon3}{0.2}
    \]
    By another application of (Slide) and (Change of Basis) we arrive at a diagram of the form 
    \[
        \skein{/skein_figs/dec_3gon4}{0.2}
    \]
    During this last step, we pick up colored strands between the black ``spokes''; one may happily move these out of the diagram.
\end{proof}


\begin{proof}[Proof of Proposition~\ref{prop:eval-criteria}]
    Begin by applying relations from the underlying trivalent category's evaluation algorithm; this decreases the number of trivalent vertices.
    What remains is a trivalent diagram with colored strand decorations.
    By a well-known Euler characteristic calculation, there must remain some black $n$-gon with $n\in\{ 2,\dots,5 \}$.
    Choose one such face and apply Lemma~\ref{lem:ext-dec} to reduce it to a singly-externally-decorated $n$-gon.
    Now one of the relations (decBigon), (decTrigon), (decTetragon), or (decPentagon) allows us to pop the face.
    This process decreases the number of faces (ignoring colored strands) in diagrams by at least 1 at every step.
    When only two faces remain we use the bigon relation to obtain a loop; apply (decStick) and the underlying loop relation to obtain a scalar.
    \end{proof}

For each quantum subgroup we construct, we will find planar algebras satisfying the conditions of Proposition~\ref{prop:eval-criteria}, and thus will know the planar algebras are evaluable. 

