\section{Cyclic extensions}\label{sec:skein}

The goal of this section is to develop the tools needed to prove evaluability of general $\Z_n$-like extensions of trivalent categories.
We expect this class of extensions to be helpful in the search for novel categories.
For example, there is work underway by the present author and Edie-Michell to use the techniques of this paper to 
construct a class of examples of {\it near-group} categories, as defined in \cite{gannon_near-groups}.
This work on near-group categories extends an underlying $SO(3)_q$ trivalent skein theory.
The present author has also computed extensions for two categories of type $SP(4)_q$, which, despite its skein theory being
generated by a braid, is of the same essence. 

This all begs the question of which leaves on the ``tree of life'' of \cite{tricats} 
might bear more fruit of this variety.
Already we have extended both categories ($SO(3)_q$ and $Fib$) covered by \cite[Theorem A]{tricats} by group-like objects.
This paper deals with all but one of the categories covered by \cite[Theorem B]{tricats}.
The categories one might next attempt such an extension of include:
\begin{itemize}
    \item The remaining category $ABA$ of \cite[Theorem B]{tricats}
    \item The category $H_3$ of \cite[Theorem C]{tricats}
\end{itemize}

General methods for demonstrating evaluability of a skein theory involve identifying some measure of complexity for a closed diagram, then showing the known relations allow one to strictly decrease this measure. 
For our underlying trivalent categories, {\it Euler-evaluability} allows us to decrement one measure of complexity: number of internal faces. 
With the new strand type, we have another measure: number of red strands. 
The underlying trivalent categories we deal with have evaluation algorithms based on the standard Euler characteristic argument.
One way to capture this evaluability is by considering dimensions of box spaces.
\begin{definition}
    In a trivalent category we define a {\bf box space} $B(k,f)$ to be the span of diagrams $k\to0$ with $f$ internal faces.
    If $\CC$ is a trivalent category such that, for $k=1,\dots,5$, the containment
    \[
    B(k,1) \subseteq B(k,0)
    \]
    holds, we will refer to $\CC$ as {\bf Euler-evaluable}.
\end{definition}





Diagrams inside a $\Z_n$-like extension exhibit the following nice properties, which will be key in proving their evaluability.
Essentially, we use the following lemmas to exchange decorated faces for singly-externally-decorated faces.
The defining relations for a $\Z_n$-like extension then pop the singly-decorated faces.

\begin{lemma}
    (1) ($\Z_n$) follows from (Recouple) and (Reverse).

    (2) (Split) follows from (Recouple) and ($\Z_n$).

    \begin{equation*}\tag{$\Z_n$}
        \underbrace{ \skein{/skein_figs/n_g_strands}{0.08} }_{\text{$n$}} 
        = z \skein{/skein_figs/dec_tet_3}{0.2}
    \end{equation*}

    \begin{equation*}\tag{Split}
        \underbrace{ \skein{/skein_figs/split_LHS}{0.125} }_{\text{$n$}} 
        = s \underbrace{ \skein{/skein_figs/split_RHS}{0.085} }_{\text{$n$}}
    \end{equation*}

\end{lemma}
% \begin{proof}
%     (1) 

%     (2) 
% \end{proof}

\begin{remark}\label{rem:both-orientations}
    The previous lemma implies that, upon reversing the orientations of the lefthand sides of the relations in 
    Definition~\ref{def:zn-ext} will give similar relations.
    This fact will be used in the proof of Lemma~\ref{lem:ext-dec}.
\end{remark}

\begin{remark}
    It is worth noting the following standard abuse of language. 
    A diagrammatically presented category such as a cyclic extension has hom-spaces which are formal spans of diagrams.
    When applying a relation such as (decTrigon) locally, the result is a {\it linear combination} of diagrams.
    Usually, though, this linear combination has some desirable quality, such as a smaller number of internal faces in each summand.
    In this instance, we prefer to say something along the lines of, 
    ``applying (decTrigon) decreases the number of internal faces,''
    instead of, for instance, the more wordy,
    ``applying (decTrigon) turns this diagram into a linear combination of diagrams with fewer internal faces.''
\end{remark}


\begin{lemma}
    In a $\Z_n$-like extension, there exist $n$ scalars $s_i$ such that the following relation holds:
    \begin{equation*}\tag{Slide}
        \skein{/skein_figs/slide_LHS}{0.1} 
        = \sum_{i=0}^{n-1} s_i \skein{/skein_figs/slide_RHS2}{0.1}^i
    \end{equation*}
\end{lemma}
\begin{proof}
    Apply (decStick) to the undecorated lower leg of the trivalent vertex to conjure a down-oriented red strand.
    Then use (Recouple), and (Change of Basis).
\end{proof}

We use this fact in the proof of the following lemma.

\begin{lemma}\label{lem:ext-dec}
    A decorated {\bf boxy} diagram in a $\Z_n$-like extension may be expressed as a $\C$-linear combination 
    of singly-externally decorated diagrams
\end{lemma}

\begin{proof}
    We prove the lemma for a decorated trigon, and leave the remaining cases to the reader. 
    We begin with a maximally-decorated trigon.
    All less decorated cases are absorbed along the way in this analysis.
    Additionally, following Remark~\ref{rem:both-orientations} we know that analogues of the defining relations 
    for a cyclic extension hold for both strand orientations of the diagrams on the lefthand side.
    Hence we begin with a diagram whose red strands are unoriented;
    this is possible by applying whichever version of the relations we need at the time.
    We may omit the labels $1,\dots,n-1$ for the red strands by the same reasoning:
    analogous relations hold for multiple red strands.

    Now, a maximally-decorated trigon is of the form:
    \[
        \skein{/skein_figs/dec_3gon1}{0.15}
    \]
    with any orientation on the red strands. We apply the relations (Swap) and (Slide) on the internal red strands to obtain a combination of diagrams of the form
    \[
        \skein{/skein_figs/dec_3gon2}{0.15}
    \]
    Now apply (Change of Basis) to reduce to a combination of diagrams of the form
    \[
        \skein{/skein_figs/dec_3gon3}{0.15}
    \]
    By another application of (Slide) and (Change of Basis) we arrive at a diagram of the form 
    \[
        \skein{/skein_figs/dec_3gon4}{0.15}
    \]
    During this last step, we pick up red strands between the black ``spokes''; one may happily move these out of the diagram.
\end{proof}




One more lemma will complete our ability to evaluate closed diagrams in $\Z_n$-like extensions.
\begin{lemma}\label{lem:decorated-graph}
    Suppose a diagram $\EE$ in a $\Z_n$-like extension consists only of black loops and 
    red oriented edges between them. 
    Suppose furthermore that each loop of $\EE$ has either exactly $n$ strands entering or exactly $n$ strands leaving.
    Then the diagram $\EE$ evaluates to a scalar.
\end{lemma}

\begin{proof}
    First note that any oriented edges starting and ending from the same black loop may be removed
    using (Swap) and (decStick).
    So assume there are only oriented edges between distinct black loops.
    We'll use graph theoretic language, with black loops playing the role of nodes,
    and oriented red edges playing the role of oriented edges.

    If a node has exactly one neighbor, use ($\Z_n$) to remove both for a $\delta^2$.
    So assume every node has at least two neighbors.
    Pick one node and call it $u$.
    Without loss of generality, assume the only oriented edges leaving $u$ are outgoing.
    Call one of its neighbors $w$;
    since $u$ has multiple neighbors, we have $\deg(u\to w)=k<n$.
    Recalling briefly that $u$ is actually a loop, we may traverse it counterclockwise
    beginning at the outgoing edge to $w$.
    Along the way there will be $n-k$ more outgoing red edges;
    each corresponds to a neighbor of $u$.
    With this counterclockwise orientation, call the remaining neighbors of $u$ by $v_1,\dots,v_{n-k}$, noting that
    these need not be distinct.
    We may label the neighbors of $w$ in a similar way, but traversing clockwise
    beginning at the incoming edge from $u$.
    Call these neighbors of $w$ by $w_1,\dots,w_{n-k}$, again noting that 
    they need not be distinct.

    The diagram is planar, so we may isotope it to look, locally, as follows:
    \begin{center}
        \begin{tikzpicture}
            \node[shape=circle,draw=black] (u) at (6,0) {$u$};

            \node[shape=circle,draw=black] (v1) at (5,1.5) {$v_1$};
            \node[shape=circle,draw=black] (v2) at (4,1.5) {$v_2$};
            \node[shape=circle,draw=black] (vnk) at (2,1.5) {$v_{n-k}$};

            \node[shape=circle,draw=black] (w) at (7,1.5) {$w$};
            \node[shape=circle,draw=black] (w1) at (7,3) {$w_1$};
            \node[shape=circle,draw=black] (w2) at (8,4) {$w_2$};
            \node[shape=circle,draw=black] (wnk) at (10,6) {$w_{n-k}$};

            \node[] (vdots) at (3,1.5) {$\dots$};
            \node[rotate=45] (wdots) at (9,5) {$\dots$};

            \path [->, draw=orange] (u) edge node {} (v1);
            \path [->, draw=orange] (u) edge node {} (v2);
            \path [->, draw=orange] (u) edge node {} (vnk);

            \path [->, draw=orange] (u) edge node [right] {$k$} (w);
 
            \path [->, draw=orange] (w1) edge node {} (w);
            \path [->, draw=orange] (w2) edge [bend left] node {} (w);
            \path [->, draw=orange] (wnk) edge [bend left] node {} (w);
        \end{tikzpicture}
    \end{center}
    Note that we may have omitted edges here.
    That is, we may have $\deg(v_i \to w_j) \neq 0$, or there may be other nodes not pictured.
    This is not an issue for us.

    Now apply (Recouple), exchanging pairs of edges $u\to v_i$ and $w_j \to w$ for pairs of edges $u\to w$ and $w_j\to v_i$.
    This changes $\deg(u\to w)$ to $n$, allowing us, using ($\Z_n$), to exchange a pair of nodes for a scalar.
    Continue until only pairs of nodes remain, exchanging each pair for a $\delta^2$.
\end{proof}







\begin{proposition}\label{prop:eval-criteria}
    A $\Z_n$-like extension of an Euler-evaluable trivalent category is evaluable.
\end{proposition}

\begin{proof}
    Suppose we begin with a diagram given by a closed, decorated planar trivalent graph.
    Begin by applying relations from the underlying trivalent category's evaluation algorithm to any undecorated faces; 
    this decreases the number of trivalent vertices.
    By the standard Euler characteristic calculation, there must remain some black $m$-gon with $m\in\{ 0,1,\dots,5 \}$.
    The $m=0$ case is handled at the end of this proof.
    The $m=1$ case is handled by (Change of Basis) and (Schur 0). 
    Choose one such face and apply Lemma~\ref{lem:ext-dec} to reduce it to a singly-externally-decorated $m$-gon.
    Now one of the relations (decBigon), (decTrigon), (decTetragon), or (decPentagon) allows us to pop the face.
    This process decreases the number of faces (ignoring red strands) in diagrams by at least 1 at every step,
    but also may increase the number of connected components in any summand.
    Continue this process until only decorated loops, or decorated loops connected by red strands remain.
    If only decorated loops remain, apply (decStick).
    
    Our diagram now consists of a number of black loops, connected by red strands.
    Use (Recouple) and ($\Z_n$) to make it so every black loop has either only in-strands or only out-strands attached to it.
    Now, if any black loop has more or fewer than $n$ strands entering or exiting it,
    then (Split), ($\Z_n$), and (Schur 0) imply the whole diagram is zero.
    So suppose each black loop has exactly $n$ strands entering or exiting.
    Apply Lemma~\ref{lem:decorated-graph} to evaluate the remaining graph for a scalar.
\end{proof}

As with all evaluability arguments, if we have nontriviality, we immediately deduce simplicity of the unit.

\begin{corollary}\label{cor:Zn-simple-unit}
    For a $\Z_n$-like extension $\DD$ of an Euler-evaluable trivalent category, we have
    \[
        \dim\Hom_{\DD} (0\to 0) \leq 1.
    \]
\end{corollary}

For each quantum subgroup we construct, we will find planar algebras satisfying the conditions of 
Proposition~\ref{prop:eval-criteria}, and thus will know the planar algebras are evaluable. 



