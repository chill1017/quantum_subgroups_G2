\section{Skein Theory Generalities}\label{sec:skein}

The goal of this section is to develop the tools needed to prove evaluability in a certain class of extensions of trivalent categories.
Namely we define the concept of a {\it $\Z_n$-like extension} of an {\it Euler-evaluable} trivalent category.
We expect this class of extensions to be helpful in the search for novel categories.
For example, there is work underway by the present author and Edie-Michell to use the techniques of this paper to 
construct a possibly infinitely large class of examples of {\it near-group} categories \cite{gannon_near-groups}.
This work on near-group categories \cite{cain_me_near_groups} extends an underlying $SO(3)_q$ trivalent skein theory.
The present author has also begun work \cite{me_sp4} on a family of extensions of $SP(4)_q$, which, despite its skein theory being
generated by a braid, is of the same essence. 


This all begs the question of which leaves on the ``tree of life'' \cite{tricats} may be extended in this way.
We expect this tree to bear much fruit of this variety.
Already we have extended both categories ($SO(3)_q$ and $Fib$) covered by \cite[Theorem A]{tricats} by group-like objects.
This paper deals with all but one of the categories covered by \cite[Theorem B]{tricats}.
The categories one might next attempt such an extension of include:
\begin{itemize}
    \item The remaining category $ABA$ of \cite[Theorem B]{tricats}
    \item The category $H_3$ of \cite[Theorem C]{tricats}
\end{itemize}

Now let us get to work defining the class of extensions we've been alluding to.
Both of the categories studied in this paper are extensions of trivalent categories by a colored, directed, $\Z_n$-like strand. 
The trivalent categories we deal with have evaluation algorithms based on the standard Euler characteristic argument.
One way to capture this evaluability is by considering dimensions of box spaces.
\begin{definition}
    In a trivalent category we define a {\bf box space} $B(k,f)$ to be the span of diagrams $k\to0$ with $f$ internal faces.
    If $\CC$ is a trivalent catgeory such that, for $k=1,\dots,5$, the constraint
    \[
    \dim B(k,1) \leq \dim B(k,0)
    \]
    holds, we will refer to $\CC$ as {\bf Euler-evaluable}.
\end{definition}

General methods for demonstrating evaluability of a skein theory involve identifying some measure of complexity for a closed diagram, then showing the known relations allow one to strictly decrease this measure. 
For our underlying trivalent categories, Euler-evaluability allows us to decrement one measure of complexity: number of internal faces. 
With the new strand type, we have another measure: number of colored strands. 
% We concatenate face count with colored strand count to obtain a lexicographic measure of complexity. 

The following notation will be useful in writing down relations in $\Z_n$-like extensions.
\begin{definition}
    Suppose a diagram $E$ has $m$ boundary points. We define $dec_i(E)$ to be the $i$-th external single clockwise decorations of $E$. For example,
    \[
        dec_1\left(\skein{/skein_figs/trivalent}{0.15}\right) = \skein{/skein_figs/rightDown}{0.1},
    \quad
    \text{and} 
    \quad
        \sum_{i=1}^{m} dec_i\left(\skein{/skein_figs/trivalent}{0.15}\right) = \skein{/skein_figs/rightDown}{0.1} +\skein{/skein_figs/bottomLeft}{0.1} +\skein{/skein_figs/leftUp}{0.1}
    \]
    We adopt the convention that $dec_0 (E) = E$.
\end{definition}

Finally we define the class of categories we will be working with, and show they are evaluable in general.
\begin{definition}\label{def:zn-ext}
    Let $\CC = \left\langle \skein{/skein_figs/trivalent}{0.15} \right\rangle$ 
    be a trivalent category that is Euler-evaluable. 
    Call $\DD$ a {\bf $\Z_n$-like extension} of $\CC$ if 
    $\DD = \left\langle \skein{/skein_figs/trivalent}{0.15}, \skein{/skein_figs/Pg}{0.15} \right\rangle$ 
    and enjoys the following relations\footnote{Conditions from \cite{czenky}.}:
    \begin{equation*}\tag{Recouple}
        \skein{/skein_figs/recouple_LHS}{0.15} 
        = c \skein{/skein_figs/recouple_RHS}{0.15}
    \end{equation*}

    \begin{equation*}\tag{Order $n$}
        \underbrace{ \skein{/skein_figs/n_g_strands}{0.10} }_{\text{$n$}} 
        = o \skein{/skein_figs/capcup}{0.15}
    \end{equation*}

    \begin{equation*}\tag{Split}
        \underbrace{ \skein{/skein_figs/split_LHS}{0.175} }_{\text{$n$}} 
        = s \underbrace{ \skein{/skein_figs/split_RHS}{0.135} }_{\text{$n$}}
    \end{equation*}

    \begin{equation*}\tag{Schur 0}
        \skein{/skein_figs/schur_0_1}{0.1} = 0 \quad 
        \skein{/skein_figs/schur_0_2}{0.1} = 0 \quad \cdots \quad 
        \underbrace{ \skein{/skein_figs/schur_0_n-1}{0.11} }_{\text{$n-1$}} = 0
    \end{equation*}

    \begin{equation*}\tag{Schur 1}
        \skein{/skein_figs/schur_1_1}{0.1} = 0 \quad 
        \skein{/skein_figs/schur_1_2}{0.1} = 0 \quad \cdots \quad 
        \underbrace{ \skein{/skein_figs/schur_1_n-1}{0.11} }_{\text{$n-1$}} = 0
    \end{equation*}

    \begin{equation*}\tag{Change of Basis}
        \skein{/skein_figs/leftUp}{0.1} 
        = \sum_{i=0}^{n-1} r_i \skein{/skein_figs/rightUp}{0.1}^i
    \end{equation*}

    \begin{equation*}\tag{Swap}
        \skein{/skein_figs/swap_LHS}{0.15} 
        = \omega \skein{/skein_figs/swap_RHS}{0.15}
    \end{equation*}

    \begin{equation*}\tag{decStick}
        \skein{/skein_figs/stick_w_Pg}{0.25} 
        = a \skein{/skein_figs/stick}{0.1}
    \end{equation*}

    \begin{equation*}\tag{decBigon}
        \skein{/skein_figs/g_bigon2}{0.1} = b \skein{/skein_figs/stick}{0.15}
    \end{equation*}

    \begin{equation*}\tag{Slide}
        \skein{/skein_figs/slide_LHS}{0.1} 
        = \sum_{i=0}^{n-1} s_i \skein{/skein_figs/slide_RHS2}{0.1}^i
    \end{equation*}

    \begin{equation*}\tag{decTrigon}
        \skein{/skein_figs/trigon_singly}{0.1} 
        = \sum_{i=0}^{n-1} t_i \skein{/skein_figs/rightUp}{0.1}^i
    \end{equation*}\label{eq:decTrigon}

    \begin{equation*}\tag{decTetragon}
         \skein{/skein_figs/square_singly}{0.12} 
         = \sum_{i=0}^4 \sum_{j=0}^3 u_{i,j} dec_i\left( r^j\left( \skein{/skein_figs/pent_1}{0.2} \right) \right)
    \end{equation*}

     \begin{equation*}\tag{decPentagon}
        \skein{/skein_figs/pentagon_singly}{0.17} 
        = \sum_{i=0}^5 \sum_{j=0}^4 v_{i,j} dec_i\left( r^j\left( \skein{/skein_figs/pent_1}{0.2} \right) \right) 
        + \sum_{i=0}^5 \sum_{j=0}^4 w_{i,j} dec_i\left( r^j\left( \skein{/skein_figs/pent_2}{0.12} \right) \right)
    \end{equation*}
\end{definition}


\begin{remark}
    A quick sketch shows that using (Order) followed by repeated applications of (Recouple) and (decStick) 
    allows one to swap an up-oriented strand for $n-1$ down-oriented strand. 
    This means that, upon reversing the orientations of the lefthand sides of the relations in 
    Definition~\ref{def:zn-ext} will give similar relations.
    This fact will be used in the proof of Lemma~\ref{lem:ext-dec}.
\end{remark}

\begin{remark}
    It is worth noting the following standard abuse of language. 
    A diagrammatically presented category such as a $\Z_n$-like extension has hom-spaces which are formal spans of diagrams.
    When applying a relation such as (decTrigon) locally, the result is clearly a {\it linear combination} of diagrams.
    Usually, though, this linear combination has some desirable quality, such as a smaller number of internal faces in each summand.
    In this instance, we prefer to say something along the lines of, 
    ``applying (decTrigon) decreases the number of internal faces,''
    instead of, for instance, the more wordy,
    ``applying (decTrigon) turns this diagram into a linear combination of diagrams with fewer internal faces.''
\end{remark}


Diagrams inside a $\Z_n$-like extension exhibit the following nice property, which will be key in proving their evaluability.
Essentially, we use the following lemma to exchange decorated faces for singly-externally-decorated faces.
The defining relations for a $\Z_n$-like extension then pop the singly-decorated faces.

\begin{lemma}\label{lem:ext-dec}
    A decorated diagram may be expressed as a combination of externally decorated diagrams
\end{lemma}
\begin{proof}
    We prove the lemma for a decorated trigon, and leave the remaining cases to the reader. 
    We begin with a maximally-decorated trigon.
    All less decorated cases are absorbed along the way in this analysis.
    Now, a maximally-decorated trigon is of the form:
    \[
        \skein{/skein_figs/dec_3gon1}{0.2}
    \]
    with any labeling on the colored strands. We apply the relations (Swap) and (Slide) on the internal colored strands to obtain a combination of diagrams of the form
    \[
        \skein{/skein_figs/dec_3gon2}{0.2}
    \]
    Now apply (Change of Basis) to reduce to a combination of diagrams of the form
    \[
        \skein{/skein_figs/dec_3gon3}{0.2}
    \]
    By another application of (Slide) and (Change of Basis) we arrive at a diagram of the form 
    \[
        \skein{/skein_figs/dec_3gon4}{0.2}
    \]
    During this last step, we pick up colored strands between the black ``spokes''; one may happily move these out of the diagram.
\end{proof}





\begin{lemma}\label{lem:decorated-graph}
    A planar diagram $\EE$ consists only of black loops and colored oriented edges between them,
    such that (Order $n$) and (Recouple) hold. 
    Suppose furthermore that each loop of $\EE$ has either exactly $n$ strands or exactly $n$ strands leaving.
    Then the diagram $\EE$ evaluates to a scalar.
\end{lemma}

\begin{proof}
    Firt note that any oriented edges starting and ending from the same black loop may be removed
    using (Swap) and (decStick).
    So assume there are only oriented edges between distinct black loops.
    We'll use graph theoretic language, with black loops playing the role of nodes,
    and oriented edges playing the role of, well, orinted edges.

    If a node has exactly one neighbor, use (Order $n$) to remove both.
    So assume every node has at least two neighbors.
    Pick one node and call it $u$.
    Choose an orintation for its neighbors.
    Call the rightmost neighbor by $w$;
    assume $\deg(u\to w)=k<n$.
    From right to left, call the remaining neighbors by $v_1,\dots,v_{n-k}$, noting that
    these need not be distinct.
    From left to right, call the neighbors of $w$ by $w_1,\dots,w_{n-k}$, again noting that 
    these need not be distinct.

    The diagram is planar, so without loss, we may isotope it to look, locally, like
    \begin{center}
        \begin{tikzpicture}
            \node[shape=circle,draw=black] (u) at (6,0) {$u$};

            \node[shape=circle,draw=black] (v1) at (5,1.5) {$v_1$};
            \node[shape=circle,draw=black] (v2) at (4,1.5) {$v_2$};
            \node[shape=circle,draw=black] (vnk) at (2,1.5) {$v_{n-k}$};

            \node[shape=circle,draw=black] (w) at (7,1.5) {$w$};
            \node[shape=circle,draw=black] (w1) at (7,3) {$w_1$};
            \node[shape=circle,draw=black] (w2) at (8,4) {$w_2$};
            \node[shape=circle,draw=black] (wnk) at (10,6) {$w_{n-k}$};

            \node[] (vdots) at (3,1.5) {$\dots$};
            \node[rotate=45] (wdots) at (9,5) {$\dots$};

            \path [->, draw=orange] (u) edge node {} (v1);
            \path [->, draw=orange] (u) edge node {} (v2);
            \path [->, draw=orange] (u) edge node {} (vnk);

            \path [->, draw=orange] (u) edge node [right] {$k$} (w);
 
            \path [->, draw=orange] (w1) edge node {} (w);
            \path [->, draw=orange] (w2) edge [bend left] node {} (w);
            \path [->, draw=orange] (wnk) edge [bend left] node {} (w);
        \end{tikzpicture}
    \end{center}

    Now apply (Recouple), exchanging pairs of edges $u\to v_i$ and $w_i \to w$ for pairs of edges $u\to w$ and $w_i\to v_i$.
    This changes $\deg(u\to w)$ to $n$, allowing us, using (Order), to exchange a pair of nodes for a scalar.
    Continue ad nauseum.
\end{proof}







\begin{proposition}\label{prop:eval-criteria}
    A $\Z_n$-like extension of an Euler-evaluable trivalent category is evaluable.
\end{proposition}

\begin{proof}
    Suppose we begin with a diagram given by a closed, decorated planar trivalent graph.
    Begin by applying relations from the underlying trivalent category's evaluation algorithm to any undecorated faces; 
    this decreases the number of trivalent vertices.
    By the standard Euler characteristic calculation, there must remain some black $n$-gon with $n\in\{ 2,\dots,5 \}$.
    Choose one such face and apply Lemma~\ref{lem:ext-dec} to reduce it to a singly-externally-decorated $n$-gon.
    Now one of the relations (decBigon), (decTrigon), (decTetragon), or (decPentagon) allows us to pop the face.
    This process decreases the number of faces (ignoring colored strands) in diagrams by at least 1 at every step,
    but also may increase the number of connected components in any summand.
    Continue this process until only decorated loops, or decorated loops connected by colored strands remain.
    If only decorated loops remain, apply (decStick).
    
    Our diagram now consists of a number of black loops, connected by colored strands.
    Use (Recouple) and (Order $n$) to make it so every black loop has either only in-strands or only out-strands attached to it.
    If any black loop has more or less than $n$ strands entering or exiting (Schur 0) implies the whole diagram is zero.
    So suppose each black loop has exactly $n$ strands entering or exiting.
    Apply Lemma~\ref{lem:decorated-graph} to evaluate the remaining graph for a scalar.
\end{proof}

For each quantum subgroup we construct, we will find planar algebras satisfying the conditions of Proposition~\ref{prop:eval-criteria}, and thus will know the planar algebras are evaluable. 



