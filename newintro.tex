\section{Introduction}\label{sec:intro}
Quantum subgroups are a well-known source of tensor categories.
More precisely, given a conformal embedding $\VV(\gg,k) \subseteq \VV(\hh,1)$ of VOAs 
as in \cite{DMNO}, one obtains a corresponding Etale algebra $A$ in $\ol{\Rep(U_{q_k}(\gg))}$.
This algebra then allows one to consider the category $\ol{\Rep(U_{q_k}(\gg))}_A$ of right $A$-modules.
A half-braiding on $A$ then gives a tensor product on $\ol{\Rep(U_{q_k}(\gg))}_A$, and one may study this
new category in its own right.
The free functor gives an embedding $\ol{\Rep(U_{q_k}(\gg))} \hookrightarrow \ol{\Rep(U_{q_k}(\gg))}_A$.
As this embedding is, in general, not full, it remains only to 
find a description of the new morphisms in $\ol{\Rep(U_{q_k}(\gg))}_A$ to describe this newly constructed category of modules.
Recent works of Edie-Michell and Snyder \cite{cain_noah} have used this reasoning, and 
representation theoretic techniques to give diagrammatic descriptions of tensor categories of modules
corresponding to the family of conformal embeddings $\VV(\sl_N,N) \subseteq \VV(\sl_{N^2-1},1)$.

On the other hand, one may start with a known category $\CC$ compute an embedding 
\[
    \CC \hookrightarrow \DD
\]
for $\DD$ some category in which explicit calculations are more easily performed by computers.
One option for $\DD$ is the graph planar algebra $\GPA(\Gamma)$ for some graph $\Gamma$.
By the GPA Embedding Theorem \cite[Theorem~\red{TODO}]{extended_haagerup}, an embedding
\[
    \CC \hookrightarrow \GPA(\Gamma)
\]
induces a module category for $\CC$.
This has been done for $\ol{\Rep(U_q(\sl_N))}$ in \cite{Cain_Dan} and for the 
extended Haagerup categories in \cite{extended_haagerup}.

The present work describes a blend of these two techniques.
We begin by finding an embedding of the well-known trivalent category $\GG_2(q)$ of \cite{Kuperberg,tricats}
into the graph planar algebra on the fusion graph of the object $V_{\Lambda_1}$.
The category $\GG_2(q)$ is known to be a diagrammatic presentation for $\ol{\Rep(U_q(\gg_2))}$.
Through the free functor we can view our GPA embedding as a GPA embedding for a $\otimes$-generating 
object's planar algebra in $\ol{\Rep(U_q(\gg_2))}_A$.
Diagrammatically, this gives us a black-strand and a trivalent vertex:
\[
    \skein{/skein_figs/trivalent}{0.14}
\]
with a known skein theory.
We then search inside the GPA embedding for new morphisms.
According to \cite{DMNO} there ought to be a projection onto a $\Z_k$-like simple object in $\ol{\Rep(U_q(\gg_2))}_A$,
so this is what we search for inside the GPA.
We view this new morphism as an I with an oriented orange vertical strand:
\[
    \skein{/skein_figs/dec_tet_5}{0.15}       
\]
In the case of $\GG_2(q)$, the properties of this new morphism were unknown beyond a few basics deriving from, e.g.,
the fact that is is a projection onto a simple object contained in the tensor square of the $\otimes$-generating object.
Once we have or hands on the image in the GPA of this projection, though, 
we may explore its properties through explicit computations.
We perform this process of extending GPA embeddings for the two conformal embeddings
\begin{equation}\label{eq:conf-embs}
    \VV(\gg_2,3) \subseteq \VV(\ee_6,1) \quad\text{and}\quad \VV(\gg_2,4) \subseteq \VV(\dd_7,1).
\end{equation}
By \ref{gannon2025_exotic_q_subgroups_extensions} there are no other integer levels 
at which $\ol{\Rep(U_{q_k}(\gg_s))}$ admits conformal embeddings.

Now we begin by introducing some notation for a skein theory involving an oriented, colored strand 
in addition to unoriented black strands.
\begin{definition}
    For a diagram $\EE$ the notation $\rho^i(\EE)$ means an $i$-click right rotation. 
    For instance, 
    \[
    \rho^1\left( \skein{/skein_figs/dec_tet_4}{0.1} \right) = \skein{/skein_figs/dec_tet_3}{0.1} 
    \quad\text{and}\quad 
    \rho^2\left( \skein{/skein_figs/dec_tet_4}{0.1} \right) = \skein{/skein_figs/dec_tet_4}{0.1}.
    \] 

    Suppose the diagram $\EE$ has $m$ boundary points. 
    We define $dec_i(\EE)$ to be the $i$-th external single clockwise decoration of $\EE$. 
    For example,
    \[
        dec_1\left(\skein{/skein_figs/trivalent}{0.1}\right) = \skein{/skein_figs/triv_rightDown}{0.1},
    \quad
    \text{and} 
    \quad
        \sum_{i=1}^{3} dec_i\left(\skein{/skein_figs/trivalent}{0.1}\right) 
        = \skein{/skein_figs/triv_rightDown}{0.1} 
        +v\skein{/skein_figs/triv_bottomLeft}{0.1} 
        +v\skein{/skein_figs/triv_leftUp}{0.1}
    \]
    We adopt the convention that $dec_0 (\EE) = \EE$.
\end{definition}

Both of the categories studied in this paper are extensions of trivalent categories by a colored, directed, $\Z_n$-like strand. 
We define the class of categories we will be working with.
Later we will show that, with a relatively tame assumption on the underlying skein theory, 
categories in this class are evaluable in general.
\begin{definition}\label{def:zn-ext}
    Let $\CC = \left\langle \skein{/skein_figs/trivalent}{0.1} \right\rangle$ 
    be a trivalent category. 
    Call $\DD$ a {\bf $\Z_n$-like extension} (or {\bf cyclic} when $n$ is understood) of $\CC$ if we have
    $\DD = \left\langle \skein{/skein_figs/trivalent}{0.1}, \skein{/skein_figs/dec_tet_5}{0.1} \right\rangle$,
    enjoying the following relations % \footnote{Conditions from \cite{czenky}.}:
    \begin{equation*}\tag{Recouple}
        \skein{/skein_figs/recouple_LHS}{0.1} 
        = c \skein{/skein_figs/recouple_RHS}{0.1}
    \end{equation*}

    \begin{equation*}\tag{\red{Reverse}}
        \skein{/skein_figs/recouple_LHS}{0.1} 
        = c \skein{/skein_figs/recouple_RHS}{0.1}
    \end{equation*}

    \begin{equation*}\tag{Schur 0}
        \skein{/skein_figs/schur_0_1}{0.1} = 0 \quad 
        \skein{/skein_figs/schur_0_2}{0.1} = 0 \quad \cdots \quad 
        \underbrace{ \skein{/skein_figs/schur_0_n-1}{0.11} }_{\text{$n-1$}} = 0
    \end{equation*}

    \begin{equation*}\tag{Schur 1}
        \skein{/skein_figs/schur_1_1}{0.1} = 0 \quad 
        \skein{/skein_figs/schur_1_2}{0.1} = 0 \quad \cdots \quad 
        \underbrace{ \skein{/skein_figs/schur_1_n-1}{0.11} }_{\text{$n-1$}} = 0
    \end{equation*}

    \begin{equation*}\tag{Swap}
        \skein{/skein_figs/swap_LHS}{0.15} 
        = \omega \skein{/skein_figs/swap_RHS}{0.15}
    \end{equation*}

    \begin{equation*}\tag{decStick}
        \skein{/skein_figs/stick_w_Pg}{0.25} 
        = \skein{/skein_figs/stick}{0.1}
    \end{equation*} 

    \begin{equation*}\tag{decBigon}
        \skein{/skein_figs/g_bigon2}{0.1} = b \skein{/skein_figs/stick}{0.15}
    \end{equation*}

    \begin{equation*}\tag{Change of Basis}
        \skein{/skein_figs/triv_leftUp}{0.1} 
        = \sum_{i=0}^{n-1} r_i \skein{/skein_figs/triv_rightUp_i}{0.1}
    \end{equation*}

    \begin{equation*}\tag{decTrigon}
        \skein{/skein_figs/dec_trigon_LHS}{0.1} 
        = \sum_{i=0}^{n-1} t_i \skein{/skein_figs/triv_rightUp_i}{0.1}
    \end{equation*}\label{eq:decTrigon}
 
    \begin{equation*}\tag{decTetragon}
         \skein{/skein_figs/dec_tet_LHS}{0.1} 
         = \sum_{i=0}^4 \sum_{j=0}^3 u_{i,j} dec_i\left( \rho^j\left( \skein{/skein_figs/dec_tet_1}{0.1} \right) \right)
         + \sum_{i=0}^4 \sum_{j=0}^3 v_{i,j} dec_i\left( \rho^j\left( \skein{/skein_figs/dec_tet_4}{0.1} \right) \right)
    \end{equation*}

     \begin{equation*}\tag{decPentagon}
        \skein{/skein_figs/dec_pent_LHS}{0.1} 
        = \sum_{i=0}^5 \sum_{j=0}^4 w_{i,j} dec_i\left( \rho^j\left( \skein{/skein_figs/dec_pent_RHS1}{0.1} \right) \right) 
        + \sum_{i=0}^5 \sum_{j=0}^4 x_{i,j} dec_i\left( \rho^j\left( \skein{/skein_figs/dec_pent_RHS2}{0.1} \right) \right)
    \end{equation*}
    for $c, \omega, b, r_i, t_i, u_{i,j}, v_{i,j}, w_{i,j}, x_{i,j} \in \C$.
    We additionally enforce the condition that the diagrams $\skein{/skein_figs/triv_rightUp_i}{0.075}$ 
    for $i=0,\dots,n-1$ span the $2\to1$ hom-space, where adopt the convention that an oriented strand labelled with $i$ refers to the presence of $i$
    parallel oriented strands.
\end{definition}








\begin{definition}\label{def:D4}
    Set $q_4 = e^{\frac{2\pi i}{48}}$ and define $\DD_4$ to be the 
    cyclic extension of $\GG_2(q_4)$ with the following structure constants:\footnote{
        We omit the (decPentagon) equation here for brevity. 
        It contains 44 nonzero summands, and can be found in the attached Mathematica notebooks.
    } 
    
    \begin{equation*}
        \skein{/skein_figs/swap_LHS}{0.1} = - \skein{/skein_figs/swap_RHS}{0.1}, \qquad
        \skein{/skein_figs/g_bigon2}{0.1} = q_4+1+q_4^{-1} \skein{/skein_figs/stick}{0.1}
    \end{equation*}

    \begin{equation*}
        \skein{/skein_figs/triv_leftUp}{0.1} =  
        q_4^{-4} \skein{/skein_figs/trivalent}{0.1} 
        + q_4^{16} \skein{/skein_figs/triv_rightUp}{0.1}
    \end{equation*}


    \begin{equation*}
        \skein{/skein_figs/dec_trigon_LHS}{0.1} = 
        -\skein{/skein_figs/trivalent}{0.1} 
        -\skein{/skein_figs/triv_rightUp}{0.1}
    \end{equation*}


    \begin{align*}
         \skein{/skein_figs/dec_tet_LHS}{0.12} & = q^2 \skein{/skein_figs/dec_tet_1}{0.1} + q_4^2 \skein{/skein_figs/dec_tet_2}{0.1} 
            + \frac{q_4^{17}}{q-q^{-1}} \skein{/skein_figs/dec_tet_3}{0.1} + q_4^2 \skein{/skein_figs/dec_tet_4}{0.1} \\
        & + \frac{1+[3]_{q_4}}{q_4^4} \skein{/skein_figs/dec_tet_5}{0.1} + \frac{[2]_{q_4}}{q_4^{13}} \skein{/skein_figs/dec_tet_6}{0.1}
            + q_4^{-14} \skein{/skein_figs/dec_tet_8}{0.1} + \frac{[2]_{q_4}}{q_4^{13}} \skein{/skein_figs/dec_tet_9}{0.1}
            + (-1) \skein{/skein_figs/dec_tet_10}{0.1}
    \end{align*}


\end{definition}



One of the two primary results we give here is that $\DD_4$ is a presentation for a category of modules corresponding to the
level 4 conformal embedding of $\gg_2$.
\begin{theorem}
    There is an equivalence
    \[
        \Ab(\ol{\DD_4}) \cong \ol{\Rep(U_{q_4}(\gg_2))}_{A_4}
    \]
    where $A_4$ is the algebra object corresponding to the level-4 conformal embedding of \ref{eq:conf-embs}.
\end{theorem}
Theorem~\ref{thm:level-3} is an analogous theorem for level 3, where $\DD_3$ is defined similarly to $\DD_4$,
with structure constants given in the attached Mathematica files.



It is not clear a priori that the defining relations for, say, $\DD_4$ lead to a nontrivial tensor category.
The general undecidability of the word problem for groups offers some evidence that this question is difficult
for a typical presentation for a tensor category.
That is, one should not expect a set of relations to yield any nontriviality.
The fact that we have a nonzero GPA embedding of $\DD_4$ is what tells us that $\DD_4$ itself is nontrivial.

% \red{check (sub)(sub)section types}
The remainder of the paper is structured as follows.
Section~\ref{sec:prelim} sets up most of the theory needed, referencing that which we do not exposit here.
This includes unoriented planar algebras, unoriented graph planar algebras, 
internal algebra and module objects, and some assorted theoretical devices and results.
Section~\ref{sec:skein} then goes on to investigate some general properties of cyclic extensions.
We expect this class of categories to be of use for researchers intent on conjuring 
examples of exotic tensor categories.
In fact, in a forthcoming paper, the present author and Cain Edie-Michell diagrammatically
present a number of near-group categories as cyclic extensions of $SO(3)_q$ trivalent categories.
We demonstrate evaluability of this class of categories under a relatively tame assumption on the
underlying trivalent skein theory.
Section~\ref{sec:gpa-emb} discusses the process of arriving at GPA embeddings.
We detail the techniques used to arrive at GPA embeddigns of trivalent categories, and then
show how we extend these embeddings to cyclic extensions.
This section also explains how we use GPA embeddings to explore relations in these extensions.
This section uses examples from level 4 ($\DD_4$) due to the fact that the numbers involved are more presentable.
The process used for level 3 ($\DD_3$) was essentially identical.
Finally, in Section~\ref{sec:equivalences} we prove that the representations we've found 
are actually presentations for the respective quantum subgroups.