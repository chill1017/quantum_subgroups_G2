\section{Introduction}\label{sec:intro}
Quantum subgroups are a well-known source of tensor categories.
More precisely, given a conformal embedding $\VV(\gg,k) \subseteq \VV(\hh,1)$ of VOAs 
as in \cite{DMNO}, one obtains a corresponding Etale algebra $A$.
This algebra then allows one to consider the category $\ol{\Rep(U_q(\gg))}_A$ of right $A$-modules.
A half-braiding on $A$ then gives a tensor product on $\ol{\Rep(U_q(\gg))}_A$, and one may study this
new category in its own right.
The free functor gives an embedding $\ol{\Rep(U_q(\gg))} \hookrightarrow \ol{\Rep(U_q(\gg))}_A$.
As this embedding is, in general, not full, it remains only to 
find a description of the new morphisms in $\ol{\Rep(U_q(\gg))}_A$.
Recent works of Edie-Michell and Snyder \cite{cain_noah} have used this reasoning, and 
representation theoretic techniques to give diagrammatic descriptions of new tensor categories of modules
corresponding to the family of conformal embeddings $\VV(\sl_N,N^2) \subseteq \VV(\sl_{N^2-1},1)$.

On the other hand, one may start with a known category and compute graph planar algebra (GPA) embeddings for it.
This has been done for $\ol{\Rep(U_q(\sl_N))}$ in \cite{Cain_Dan} and for the 
extended Haagerup categories in \cite{extended_haagerup}.
This computation has the theoretical and practical consequences.
By the GPA embedding theorem \cite{something}, such an embedding immediately gives a module category.
It additionally gives a concrete representation of the category in which one may perform explicit computations.

The present work describes a blend of these two techniques.
We begin by fidning a GPA embedding on the well-known trivalent category $\GG_2(q)$ of \cite{Kuperberg,tricats}
which is a diagrammatic presentation for $\ol{\Rep(U_q(\gg_2))}$.
Through the free fucntor we can view our GPA embedding as a GPA embedding for a $\otimes$-generating 
object's planar algebra in $\ol{\Rep(U_q(\gg_2))}_A$.
This gives us a black-strand.
We then search inside the GPA embedding for new morphisms.
According to \cite{DMNO} there ought to be a projection onto a $\Z_k$-like simple object in $\ol{\Rep(U_q(\gg_2))}_A$,
so this is what we search for inside the GPA.
We view this new morphism as an orange strand.
The properties of this new morphism are unknown beyond a scant basic set.
Once we have or hands on the imagine in the GPA of this projection, though, 
we may explore its properties through explicit computations.
We perform this process of extending GPA embeddings for the two conformal embeddings
\begin{equation}\label{eq:conf-embs}
    \VV(\gg_2,3) \subseteq \VV(\ee_6,1) \quad\text{and}\quad \VV(\gg_2,4) \subseteq \VV(\dd_7,1).
\end{equation}

Now we begin by introducing some notation for a skein theory involving an oritned, colored strand 
in addition to unoriented black strands.
\begin{definition}
    For a diagram $\EE$ the notation $r^i(\EE)$ means an $i$-click right rotation. 
    For instance, 
    \[
    r^1\left( \skein{/skein_figs/doubleStick}{0.07} \right) = \skein{/skein_figs/capcup}{0.07} 
    \quad\text{and}\quad 
    r^2\left( \skein{/skein_figs/doubleStick}{0.07} \right) = \skein{/skein_figs/doubleStick}{0.07}.
    \] 

    Suppose the diagram $\EE$ has $m$ boundary points. 
    We define $dec_i(\EE)$ to be the $i$-th external single clockwise decoration of $\EE$. 
    For example,
    \[
        dec_1\left(\skein{/skein_figs/trivalent}{0.15}\right) = \skein{/skein_figs/rightDown}{0.1},
    \quad
    \text{and} 
    \quad
        \sum_{i=1}^{3} dec_i\left(\skein{/skein_figs/trivalent}{0.15}\right) 
        = \skein{/skein_figs/rightDown}{0.1} 
        +v\skein{/skein_figs/bottomLeft}{0.1} 
        +v\skein{/skein_figs/leftUp}{0.1}
    \]
    We adopt the convention that $dec_0 (\EE) = \EE$.
\end{definition}

Both of the categories studied in this paper are extensions of trivalent categories by a colored, directed, $\Z_n$-like strand. 
We define the class of categories we will be working with.
In Section~\ref{sec:skein} we will show that, with an assumption on the underlying skein theory, 
categories in this class are evaluable in general.
\begin{definition}\label{def:zn-ext}
    Let $\CC = \left\langle \skein{/skein_figs/trivalent}{0.1} \right\rangle$ 
    be a trivalent category. 
    Call $\DD$ a {\bf $\Z_n$-like extension} of $\CC$ if we have
    $\DD = \left\langle \skein{/skein_figs/trivalent}{0.1}, \skein{/skein_figs/Pg}{0.1} \right\rangle$,
    enjoying the following relations\footnote{Conditions from \cite{czenky}.}:
    \begin{equation*}\tag{Recouple}
        \skein{/skein_figs/recouple_LHS}{0.1} 
        = c \skein{/skein_figs/recouple_RHS}{0.1}
    \end{equation*}

    \begin{equation*}\tag{$\Z_n$}
        \underbrace{ \skein{/skein_figs/n_g_strands}{0.08} }_{\text{$n$}} 
        = o \skein{/skein_figs/capcup}{0.1}
    \end{equation*}

    \begin{equation*}\tag{Split}
        \underbrace{ \skein{/skein_figs/split_LHS}{0.125} }_{\text{$n$}} 
        = s \underbrace{ \skein{/skein_figs/split_RHS}{0.085} }_{\text{$n$}}
    \end{equation*}

    \begin{equation*}\tag{Schur 0}
        \skein{/skein_figs/schur_0_1}{0.1} = 0 \quad 
        \skein{/skein_figs/schur_0_2}{0.1} = 0 \quad \cdots \quad 
        \underbrace{ \skein{/skein_figs/schur_0_n-1}{0.11} }_{\text{$n-1$}} = 0
    \end{equation*}

    \begin{equation*}\tag{Schur 1}
        \skein{/skein_figs/schur_1_1}{0.1} = 0 \quad 
        \skein{/skein_figs/schur_1_2}{0.1} = 0 \quad \cdots \quad 
        \underbrace{ \skein{/skein_figs/schur_1_n-1}{0.11} }_{\text{$n-1$}} = 0
    \end{equation*}

    \begin{equation*}\tag{Swap}
        \skein{/skein_figs/swap_LHS}{0.15} 
        = \omega \skein{/skein_figs/swap_RHS}{0.15}
    \end{equation*}

    \begin{equation*}\tag{decStick}
        \skein{/skein_figs/stick_w_Pg}{0.25} 
        = a \skein{/skein_figs/stick}{0.1}
    \end{equation*}

    \begin{equation*}\tag{decBigon}
        \skein{/skein_figs/g_bigon2}{0.1} = b \skein{/skein_figs/stick}{0.15}
    \end{equation*}

    \begin{equation*}\tag{Change of Basis}
        \skein{/skein_figs/leftUp}{0.1} 
        = \sum_{i=0}^{n-1} r_i \skein{/skein_figs/rightUp}{0.1}^i
    \end{equation*}

    \begin{equation*}\tag{Slide}
        \skein{/skein_figs/slide_LHS}{0.1} 
        = \sum_{i=0}^{n-1} s_i \skein{/skein_figs/slide_RHS2}{0.1}^i
    \end{equation*}

    \begin{equation*}\tag{decTrigon}
        \skein{/skein_figs/trigon_singly}{0.1} 
        = \sum_{i=0}^{n-1} t_i \skein{/skein_figs/rightUp}{0.1}^i
    \end{equation*}\label{eq:decTrigon}

    \begin{equation*}\tag{decTetragon}
         \skein{/skein_figs/square_singly}{0.12} 
         = \sum_{i=0}^4 \sum_{j=0}^3 u_{i,j} dec_i\left( r^j\left( \skein{/skein_figs/eye_map}{0.1} \right) \right)
         + \sum_{i=0}^4 \sum_{j=0}^3 v_{i,j} dec_i\left( r^j\left( \skein{/skein_figs/doubleStick}{0.1} \right) \right)
    \end{equation*}

     \begin{equation*}\tag{decPentagon}
        \skein{/skein_figs/pentagon_singly}{0.17} 
        = \sum_{i=0}^5 \sum_{j=0}^4 w_{i,j} dec_i\left( r^j\left( \skein{/skein_figs/pent_1}{0.2} \right) \right) 
        + \sum_{i=0}^5 \sum_{j=0}^4 x_{i,j} dec_i\left( r^j\left( \skein{/skein_figs/pent_2}{0.12} \right) \right)
    \end{equation*}
\end{definition}


\begin{remark}
    A quick sketch shows that using (Order) followed by repeated applications of (Recouple) and (decStick) 
    allows one to swap an up-oriented strand for $n-1$ down-oriented strand. 
    This means that, upon reversing the orientations of the lefthand sides of the relations in 
    Definition~\ref{def:zn-ext} will give similar relations.
    This fact will be used in the proof of Lemma~\ref{lem:ext-dec}.
\end{remark}


\begin{remark}
    It is worth noting the following standard abuse of language. 
    A diagrammatically presented category such as a $\Z_n$-like extension has hom-spaces which are formal spans of diagrams.
    When applying a relation such as (decTrigon) locally, the result is clearly a {\it linear combination} of diagrams.
    Usually, though, this linear combination has some desirable quality, such as a smaller number of internal faces in each summand.
    In this instance, we prefer to say something along the lines of, 
    ``applying (decTrigon) decreases the number of internal faces,''
    instead of, for instance, the more wordy,
    ``applying (decTrigon) turns this diagram into a linear combination of diagrams with fewer internal faces.''
\end{remark}



\begin{definition}
    Set $q_4 = e^{\frac{2\pi i}{48}}$ and deifne $\DD_4$ to be the $\Z_2$-like extension of $\GG_2(q_4)$ with structure constants
    \begin{equation*}
    \omega = -1
    \end{equation*}

    \begin{equation*}
        r_1 = e^{-\frac{\pi i}{6}}, \quad r_2 = e^{-\frac{2\pi i}{3}}
    \end{equation*}

    \begin{equation*}
        s_1 = e^{-\frac{\pi i}{6}} \quad s_2 = e^{-\frac{4\pi i}{3}}
    \end{equation*}

    \begin{equation*}
        t_1 = -1 \quad t_2 = -1
    \end{equation*}

    \begin{align*}
        u = 1, && u = 2, && u = 3, && u = 4, && u = 5, && u = 6, \\
        v = 7, && v = 8, && v = 9, && v = 10, && v = 11, && v = 12 
    \end{align*}


\end{definition}



One of the two primary results we give here is that $\DD_4$ is a presentation for a category of modules corresponding to the
level 4 conformal embedding of $\gg_2$.
\begin{theorem}
    There is an equivalence
    \[
        \Ab(\DD_4) \cong \ol{\Rep(U_{q_4}(\gg_2))}_{A_4}
    \]
    where $A_4$ is the algebra object corresponding to the level-4 conformal embedding of \ref{eq:conf-embs}.
\end{theorem}
There is an analogous theorem for level 3, with structure constants given in the attached Mathematica files.

It is not clear a priori that the defining relations for, say, $\DD_4$ lead to a nontirivial tensor category.
The general undecidability of the word problem for groups offers some evidence that this question is difficult
for a typical presentation for a tensor category.
That is, one should not expect a set of relations to yield any nontriviality.
It follows that the presentations we give here are interesting and worth investigating more generally.

The remainder of the paper is structured as follows.
Section~\ref{sec:prelim} sets up most of the theory needed, referencing that which we do not exposit here.
This includes unoriented planar algebras, unoriented graph planar algebras, 
internal algebra and module objects, and some assorted theoretical devices and results.

Section~\ref{sec:skein} then goes on to investigate some properties of $\Z_n$-like extensions.
We expect this class of categories to be of use for researchers intent on conjuring 
examples of exotic tensor categories.
In fact, in a forthcoming paper, the present author and Cain Edie-Michell diagrammatically
present a number of near-group categories as $\Z_n$-like extensions of $SO(3)_q$ trivalent categories.
We demonstrate evaluability of this class of catgeories under a relatively tame assumption on the
underlying trivalent skein theory.

Section~\ref{sec:methods} discusses the process of arriving at GPA embeddings.
Subsection~\ref{subsec:triv-vertex} details the techniques used to arrive at GPA embeddigns of 
trivalent categories. 
Subsection~\ref{subsec:proj-relns} shows how we extend these embeddings of trivalent categories
to embeddings of $\Z_n$-like extensions, and how we use GPA embeddings to explore 
relations in these extensions.
This section uses examples from level 4 ($\DD_4$) due to the fact that the numbers involved are more presentable.
The process used for level 3 ($\DD_3$) was essentially identical.

Finally, Section~\ref{sec:results} gives the structure constants for the newly constructed categories.
\red{ Subsection~\ref{subsec:} discusses the argument used to prove that we truly have found full presentations. }
The full argument appears in \cite{cain_noah_hans}, and is adapted to the present settign without a problem.