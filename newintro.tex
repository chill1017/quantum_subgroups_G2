\section{Introduction}
Quantum subgroups are a well-known source of tensor categories.
More precisely, given a conformal embedding $\VV(\gg,k) \subseteq \VV(\hh,1)$ of VOAs 
as in \cite{DMNO}, one obtains a corresponding Etale algebra $A$.
This algebra then allows one to consider the category $\ol{\Rep(U_q(\gg))}_A$ of right $A$-modules.
A half-braiding on $A$ then gives a tensor product on $\ol{\Rep(U_q(\gg))}_A$, and one may study this
new category in its own right.
The free functor then gives an embedding $\ol{\Rep(U_q(\gg))} \hookrightarrow \ol{\Rep(U_q(\gg))}_A$,
and it remains only to find a description of the new morphisms in $\ol{\Rep(U_q(\gg))}_A$.
Recent works of Edie-Michell and Snyder \cite{cain_noah} have used this reasoning, and 
representation theoretic techniques to give diagrammatic descriptions of new tensor categories of modules
corresponding to the family of conformal embeddings $\VV(\sl_N,N^2) \subseteq \VV(\sl_{N^2-1},1)$.

On the other hand, one may start with a known category and compute graph planar algebra (GPA) embeddings for it.
This has been done for $\ol{\Rep(U_q(\sl_N))}$ in \cite{Cain_Dan} and for the 
extended Haagerup categories in \cite{extended_haagerup}.
This computation has the theoretical and practical consequences.
By the GPA embedding theorem \cite{something}, such an embedding immediately gives a module category.
It additionally gives a concrete representation of the category in which one may perform explicit computations.

The present work describes a blend of these two techniques.
We begin by fidning a GPA embedding on the well-known trivalent category $\GG_2(q)$ of \cite{Kuperberg,tricats}
which we view as a diagrammatic presentation for $\ol{\Rep(U_q(\gg_2))}$.
Through the free fucntor we can view our GPA embedding as a GPA embedding for a $\otimes$-generating 
object's planar algebra in $\ol{\Rep(U_q(\gg_2))}_A$.
This gives us a black-strand.
We then search inside the GPA embedding for new morphisms.
According to \cite{DMNO} there ought to be a projection onto a $\Z_k$-like simple object in $\ol{\Rep(U_q(\gg_2))}_A$,
so this is what we search for inside the GPA.
We view this new morphism as an orange strand.
The properties of this new morphism are unknown beyond a scant basic set.
Once we have or hands on the imagine in the GPA of this projection, though, 
we may explore its properties through explicit computations.
We perform this process of extending GPA embeddings for the two conformal embeddings
\begin{equation}\label{eq:conf-embs}
    \VV(\gg_2,3) \subseteq \VV(\ee_6,1) \quad\text{and}\quad \VV(\gg_2,4) \subseteq \VV(\dd_7,1).
\end{equation}

Now we begin by introducing some notation for a skein theory involving an oritned, colored strand 
in addition to unoriented black strands.
\begin{definition}
    For a diagram $\EE$ the notation $r^i(\EE)$ means an $i$-click right rotation. 
    For instance, 
    \[
    r^1\left( \skein{/skein_figs/doubleStick}{0.07} \right) = \skein{/skein_figs/capcup}{0.07} 
    \quad\text{and}\quad 
    r^2\left( \skein{/skein_figs/doubleStick}{0.07} \right) = \skein{/skein_figs/doubleStick}{0.07}.
    \] 

    Suppose the diagram $\EE$ has $m$ boundary points. 
    We define $dec_i(\EE)$ to be the $i$-th external single clockwise decoration of $\EE$. 
    For example,
    \[
        dec_1\left(\skein{/skein_figs/trivalent}{0.15}\right) = \skein{/skein_figs/rightDown}{0.1},
    \quad
    \text{and} 
    \quad
        \sum_{i=1}^{3} dec_i\left(\skein{/skein_figs/trivalent}{0.15}\right) = \skein{/skein_figs/rightDown}{0.1} +\skein{/skein_figs/bottomLeft}{0.1} +\skein{/skein_figs/leftUp}{0.1}
    \]
    We adopt the convention that $dec_0 (\EE) = \EE$.
\end{definition}

Both of the categories studied in this paper are extensions of trivalent categories by a colored, directed, $\Z_n$-like strand. 
The underlying trivalent categories we deal with have evaluation algorithms based on the standard Euler characteristic argument.
One way to capture this evaluability is by considering dimensions of box spaces.
\begin{definition}
    In a trivalent category we define a {\bf box space} $B(k,f)$ to be the span of diagrams $k\to0$ with $f$ internal faces.
    If $\CC$ is a trivalent catgeory such that, for $k=1,\dots,5$, the constraint
    \[
    \dim B(k,1) \leq \dim B(k,0)
    \]
    holds, we will refer to $\CC$ as {\bf Euler-evaluable}.
\end{definition}

Next we define the class of categories we will be working with.
In Section~\ref{sec:skein} we will show they are evaluable in general.
\begin{definition}\label{def:zn-ext}
    Let $\CC = \left\langle \skein{/skein_figs/trivalent}{0.1} \right\rangle$ 
    be a trivalent category that is Euler-evaluable. 
    Call $\DD$ a {\bf $\Z_n$-like extension} of $\CC$ if 
    $\DD = \left\langle \skein{/skein_figs/trivalent}{0.1}, \skein{/skein_figs/Pg}{0.1} \right\rangle$ 
    and $\DD$ enjoys the following relations\footnote{Conditions from \cite{czenky}.}:
    \begin{equation*}\tag{Recouple}
        \skein{/skein_figs/recouple_LHS}{0.1} 
        = c \skein{/skein_figs/recouple_RHS}{0.1}
    \end{equation*}

    \begin{equation*}\tag{$\Z_n$}
        \underbrace{ \skein{/skein_figs/n_g_strands}{0.08} }_{\text{$n$}} 
        = o \skein{/skein_figs/capcup}{0.1}
    \end{equation*}

    \begin{equation*}\tag{Split}
        \underbrace{ \skein{/skein_figs/split_LHS}{0.125} }_{\text{$n$}} 
        = s \underbrace{ \skein{/skein_figs/split_RHS}{0.085} }_{\text{$n$}}
    \end{equation*}

    \begin{equation*}\tag{Schur 0}
        \skein{/skein_figs/schur_0_1}{0.1} = 0 \quad 
        \skein{/skein_figs/schur_0_2}{0.1} = 0 \quad \cdots \quad 
        \underbrace{ \skein{/skein_figs/schur_0_n-1}{0.11} }_{\text{$n-1$}} = 0
    \end{equation*}

    \begin{equation*}\tag{Schur 1}
        \skein{/skein_figs/schur_1_1}{0.1} = 0 \quad 
        \skein{/skein_figs/schur_1_2}{0.1} = 0 \quad \cdots \quad 
        \underbrace{ \skein{/skein_figs/schur_1_n-1}{0.11} }_{\text{$n-1$}} = 0
    \end{equation*}

    \begin{equation*}\tag{Change of Basis}
        \skein{/skein_figs/leftUp}{0.1} 
        = \sum_{i=0}^{n-1} r_i \skein{/skein_figs/rightUp}{0.1}^i
    \end{equation*}

    \begin{equation*}\tag{Swap}
        \skein{/skein_figs/swap_LHS}{0.15} 
        = \omega \skein{/skein_figs/swap_RHS}{0.15}
    \end{equation*}

    \begin{equation*}\tag{decStick}
        \skein{/skein_figs/stick_w_Pg}{0.25} 
        = a \skein{/skein_figs/stick}{0.1}
    \end{equation*}

    \begin{equation*}\tag{decBigon}
        \skein{/skein_figs/g_bigon2}{0.1} = b \skein{/skein_figs/stick}{0.15}
    \end{equation*}

    \begin{equation*}\tag{Slide}
        \skein{/skein_figs/slide_LHS}{0.1} 
        = \sum_{i=0}^{n-1} s_i \skein{/skein_figs/slide_RHS2}{0.1}^i
    \end{equation*}

    \begin{equation*}\tag{decTrigon}
        \skein{/skein_figs/trigon_singly}{0.1} 
        = \sum_{i=0}^{n-1} t_i \skein{/skein_figs/rightUp}{0.1}^i
    \end{equation*}\label{eq:decTrigon}

    \begin{equation*}\tag{decTetragon \red{wrong rhs!!}}
         \skein{/skein_figs/square_singly}{0.12} 
         = \sum_{i=0}^4 \sum_{j=0}^3 u_{i,j} dec_i\left( r^j\left( \skein{/skein_figs/pent_1}{0.2} \right) \right)
    \end{equation*}

     \begin{equation*}\tag{decPentagon}
        \skein{/skein_figs/pentagon_singly}{0.17} 
        = \sum_{i=0}^5 \sum_{j=0}^4 v_{i,j} dec_i\left( r^j\left( \skein{/skein_figs/pent_1}{0.2} \right) \right) 
        + \sum_{i=0}^5 \sum_{j=0}^4 w_{i,j} dec_i\left( r^j\left( \skein{/skein_figs/pent_2}{0.12} \right) \right)
    \end{equation*}
\end{definition}


Deifne $\DD_4$ to be the $\Z_2$-like extension of $\GG_2(q_4)$ with structure constants
\[......\]
One of our primary results shows that the tensor category $\ol{\Rep(U_{q_4}(\gg_2))}_{A_4}$
corresponding to the conformal embeddings \ref{eq:conf-embs} may be presented as $\Z_2$-like extension
of $\GG_2(q_4)$.
\begin{theorem}
    There is an equivalence
    \[
        \Ab(\DD_4) \cong \ol{\Rep(U_{q_4}(\gg_2))}_{A_4}
    \]
    where $A_4$ is the algebra object corresponding to the level-4 conformal embedding of \ref{eq:conf-embs}.
\end{theorem}
We have an analogous theorem for level 3, with structure constants given in the attached Mathematica files.