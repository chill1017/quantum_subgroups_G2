\section{Methods: GPA Embeddings}

The defining relations for $\DD_3$ and $\DD_4$ were not computed theoretically. Instead, we deduced them from embedding the planar algebras $\PP_{Y_4;\cat{q_3}_{A_3}}$ and $\PP_{Y_4;\cat{q_4}_{A_4}}$ into graph planar algebras.

One may give a functor $F:\PP_{Y} \to \GPA(\Gamma)$ by giving the image of the morphism 
\[
F\left( \skein{skein_figs/trivalent}{0.08} \right) \in \Hom_{\GPA(\Gamma)}(2\to1).
\]
This amounts to giving a list of $M\coloneqq\tr(\Gamma^2\cdot\Gamma)$ complex scalars, say $a_1,\dots,a_M$. These complex numbers satisfy equations in the $a_i$ and $\ol{a_i}$. If we assume for now that each $a_i$ is real, then this reduces the system to a collection of polynomials in the $a_i$ \footnote{This assumption is useful only if it turns out to help us solve the system. In fact, any assumptions we make about this system, if they yield solutions, are in some way valid.}.
Once we have the image of the trivalent vertex in hand, we have found an embedding of the planar algebra it generates. We can then solve for the image
\[
F\left( \skein{skein_figs/Pg}{0.08} \right) \in \Hom_{\GPA(\Gamma)}(2\to2).
\]

\subsection{Trivalent Vertex}\label{subsec:triv-vertex}
The goal of this subsection is to describe in more detail the process of finding the image of the trivalent vertex. We will walk through the details for the level 4 case. The level 3 case follows the same process, but is somewhat less instructive. See Sections~\ref{} to find the exact graph used at level 4.

Let 
\[
(p_1,q_1),\dots,(p_M,q_M)
\]
be the defining basis for $\Hom_{\GPA(\Gamma)}(2\to1)$ ($M=88$ at level 4). Then it must be that 
\[
F\left( \skein{skein_figs/trivalent}{0.08} \right) = a_1(p_1,q_1)+\cdots+ a_M(p_M,q_M).
\]
The \ref{eq:Bigon} relation, when sent through $F$, becomes the system
\[
\sum_{i=1}^M a_i(p_i,q_i) \circ \sum_{j=1}^M a_j(q_j,p_j) = k^2 \sum_{e\in E(\Gamma)} (e,e).
\]
This system is quadratic in the $a_i$ since it involves up to two trivalent vertices on either side. The \ref{eq:Lolli} and \ref{eq:Rotate} relations therefore determine a system of linear equations; the others give cubic, quartic, and quintic equations. It is often useful to solve the linear subsystem first and substitute the solution into the quadratic equations. For example, at level 4, discussed in Section~\ref{sec:level-4}, we solve the linear subsyetm, substitute the solution, and isolate the following resulting equations:
\begin{align*}
    a_{8}^2+a_{85}^2 & = 4-\sqrt{2}+2 \sqrt{3}-\sqrt{6} \\
    a_{69}^2+\left(1+\sqrt{\frac{3}{2}}\right) a_{8}^2 & = \frac{3+\sqrt{3}+\sqrt{6}}{\sqrt{2}} \\
    a_{69}^2 \left(\left(2+\sqrt{6}\right) a_{8}^2+\left(2+\sqrt{6}\right)
   a_{85}^2-2 \sqrt{2+\sqrt{3}}\right) & = 5+\sqrt{2}+\sqrt{3}+2 \sqrt{6} \\ 
   2 a_{69}^4+\left(5+2 \sqrt{6}\right)
   a_{85}^4 & = \left(3+\sqrt{2}+\sqrt{3}+\sqrt{6}\right) a_{85}^2+3
   \sqrt{6}+\sqrt{3}+2 \sqrt{2}+7 \\
\end{align*}
Up to three choices of sign, the solution to this system is
\begin{align*}
    a_{8} & = \sqrt{2+\sqrt{3}-\sqrt{2+\sqrt{3}}} \\
    a_{69} & = \sqrt{\frac{1}{2} \left(-1+\sqrt{2}+\sqrt{3}\right)} \\
    a_{85} & = \sqrt{2+\sqrt{3}-\sqrt{2+\sqrt{3}}} \\
\end{align*}
Similar equations containing $a_{31}$, $a_{55}$, and $a_{63}$ appear as well. We may repeat this process and obtain the additional values
\begin{align*}
    a_{31} & = \sqrt{2+\sqrt{3}-\sqrt{2+\sqrt{3}}} \\
    a_{55} & = \sqrt{1-\sqrt{\frac{3}{2}}+\frac{1}{\sqrt{2}}} \\
    a_{63} & = \sqrt{2+\sqrt{3}-\sqrt{2+\sqrt{3}}} \\
\end{align*}
These six degree-8 algebraic numbers now begin a cascade of equation solving. They, along with the linear solution, reduce many of the original high-order equations to linear. We solve those, then repeat the process until we're forced to confront nonlinearity. The nonlinearity we encounter forces us to extract square roots, and ending up with degree-16 algebraic numbers. This lead to us concluding, for instance, that
\[
a_{10} = \frac{1}{2} \left(\sqrt{1+\sqrt{6-3 \sqrt{3}}}+\sqrt{\sqrt{2+\sqrt{3}}-1}\right).
\]
In Section~\ref{} we will discuss these numbers further.





\subsection{Projection}
The process of finding a GPA embedding of the projection $\skein{/skein_figs/Pg}{0.08}$ is similar to the process of finding the trivalent vertex described above. Now, however, we may use relations involving both $\skein{/skein_figs/trivalent}{0.08}$ and $\skein{/skein_figs/Pg}{0.08}$. Call the coordinates of the projection $b_i$. Since the coordinates of the trivalent vertex are now known, the degree of the resulting equations now depends only on the number of projection strands appearing. Moreover, we do not necessarily assume the $b_i$ are real; hence the resulting equations are polynomial in $b_i$ and $\ol{b_i}$. The following Proposition, Lemma 2.4 from \cite{cain_noah_hans}, provides a system of linear equations in the projection coefficients and is therefore of great utility.

\begin{proposition}
    The following relation holds:
    \begin{equation*}\tag{Half-braid}
        \skein{/skein_figs/hb_top}{0.15} = \skein{/skein_figs/hb_bottom}{0.2}
    \end{equation*}
\end{proposition}






\subsection{Finding Relations} When searching for a GPA embedding of the underlying trivalent category, we have a set of known relations on the trivalent vertex; i.e., the defining relations of $\GG_2(q)$. In order to uncover relations such as those of Proposition~\ref{prop:eval-criteria}, we use a process reminiscent of the scientific method. This process begins by considering a decorated trivalent graph and searching for ways to decrease its complexity. 
We begin the process by assuming no more than a minimal collection of moves; i.e., (decStick), (Recouple), and (Product).
Once we have applied all these minimal moves, we begin to barter. We look for the least \red{egregious} move we might make, and assume it comes at some cost. For example, being able to apply (Swap) might allow us to then apply (decStick); thus we eliminate a colored strand at the cost of a scalar $\omega$. This is a trade we are willing to make.
% Talk about then looking in the GPA for (Swap)
In order to find the exact price of this trade, we use our GPA-embedding $F$. We use the embedding coefficients we previously found to set up and solve the linear equation
\[
    F\left( \skein{/skein_figs/swap_LHS}{0.15} \right) = \omega F\left( \skein{/skein_figs/swap_RHS}{0.15} \right)
\]
for $\omega$.
Another example of bartering appears in the proof of Lemma~\ref{lem:ext-dec}. During the proof, we apply (Swap) and (Change of Basis) with the goal of ridding ourselves of internal strands; this comes at the cost of external strands.