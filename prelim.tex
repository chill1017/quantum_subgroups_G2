\section{Preliminaries}\label{sec:prelim}
Here we define the tools we'll use. 
This includes planar algebras, graph planar algebras, and internal algebra and module objects.
We give only a few necessary results, and refer the reader to the definitive publications.
For the general theory of tensor categories, see \cite{EGNO}.



\subsection{Algebra and Module Objects}
We will ultimately show that $\DD_3$ and $\DD_4$ are presentations for the categories
$\cat{q_3}_{A_3}$ and $\cat{q_4}_{A_4}$ of modules over algebra objects $A_3$ and $A_4$ coming from the conformal embeddings 
$\CC(\gg_2,3) \subseteq \CC(\ee_6,1)$ and $\CC(\gg_2,4) \subseteq \CC(\dd_7,1)$, respectively. 
In this subsection we recall basic facts about algebra and module objects,
as well as conformal embeddings. 
See \cite{EGNO, ostrik2001modulecategoriesweakhopf} for more complete descriptions. 
The theory which will apply to our context is given in \cite{cain_noah}.
Some basic properties concerning the interaction of algebra and module objects with
monoidal functors will be used in the proof of our main theorems;
this material can be found in \cite{monoidalFunctorsAndAlgebras}.
We restate a few definitions and facts here.
Unless otherwise stated, we will be assuming the underlying tensor categories are braided.


\begin{definition}
    Let $A$ be an algebra object of the braided tensor category $\CC$.
    $A$ is an {\bf Etale} algebra if it is commutative and separable.
    We call $A$ {\bf connected} if it is Etale and $\dim\Hom_\CC(\unit \to A)=1$.
\end{definition}


For an Etale algebra object $A$ of $\CC$, we denote by $\CC_A$ the collection of right $A$-modules internal to $\CC$.
As described in \cite{cain_noah}, a braiding on $\CC$ induces a tensor product on $\CC_A$.
Separability of $A$ implies semisimplicity of $\CC_A$, 
and connectedness of $A$ implies the unit $\unit_{\CC_A}=A$ is simple in $\CC_A$ \cite{DMNO}.

The Etale conditions are precisely those required to perform the skein theory on $A$-modules 
to define the tensor product on $\CC_A$.
Etale is the same thing as multiplication having a section, which means we can pop $A$-bigons.
Connected means theres only one $A$-cap and one $A$-cup.
This lets us wiggle enough to perform the proof that $M\otimes_A N$ is well-defined.

Furthermore, when $\CC$ is semisimple, the free functor 
\[
    \FF_A: \CC \xhookrightarrow{X\mapsto A\otimes X} \CC_A
\]
is a monoidal embedding which is, as we will see later, not always full.
Its right adjoint is the forgetful functor $\FF^\vee:\CC_A\to\CC$ 
which acts on objects $(M,\mu_M)$ by dropping the multiplication map $\mu_M: M\otimes A\to M$
and on morphisms as the identity.

Beginning with the following lemma, which is recreated from \cite{exactSequencesTensorCategories},
we now recall some facts that will help us along the way.
These will consist of a few results, along with the basics of {\it conformal embeddings}.

\begin{lemma}\label{lem:exact-functor}
    Let $\FF:\CC\to\DD$ be a monoidal functor with faithful exact right adjoint $R$. 
    If we define $A\coloneqq R(\unit)$, then there is an equivalence $\KK$ such that the diagram
    \[
    \xymatrix@C=60pt@R=45pt{
    \CC \ar[r]^{\FF} \ar[dr]^{\FF_A} & \DD \ar[d]^{\KK} \\
     & \CC_A
     }
    \]
    commutes up to natural isomorphism.
\end{lemma}


\begin{lemma}\label{lem:simple-unit-unitary}
    Suppose $\CC$ has simple unit, $\DD$ is unitary, and $\FF:\CC\to \DD$ is a $\dagger$-functor.
    Then $\ol{\CC}$ is unitary, and $F$ descends to a $\dagger$-embedding $\ol{\FF}:\ol{\CC}\to \DD$ such that
    \[
    \xymatrix@C=60pt@R=45pt{
    \CC \ar[r]^{\FF} \ar@{->>}[d]^{} & \DD \\
    \ol{\CC} \ar@{^{(}->}[ur]^{\ol{\FF}} & 
     }
    \]
    commutes.
\end{lemma}




One result which will help immensely in arriving at GPA embeddings is the following,
which is Lemma 2.4 of \cite{cain_noah}.
Since the free functor $\FF_A$ gives an embedding $\CC \hookrightarrow \CC_A$, 
the braid $c_{X,Y}$ in $\CC$ is mapped to $\FF_A(c_{X,Y})$ which defines a braided structure
for the subcategory $\FF_A(\CC)$ of $\CC_A$.
Since the free functor is in general not full, we cannot extend this to a braiding on all of $\CC_A$,
however, there is a half-braid for arbitrary morphisms between objects in the image of $\FF_A$.

\begin{lemma}[Half-braid]\label{lem:general-half-braid}
    Let $\CC$ be a braided tensor category, and $A$ an Etale algebra object.
    For any $f\in\Hom_{\CC_A}( \FF_A(Y_1) \to \FF_A(Y_2) )$, the following relation holds:
    \begin{equation*}\tag{Half-braid}
        \skein{/skein_figs/arb_hb_top}{0.2} = \skein{/skein_figs/arb_hb_bottom}{0.2}
    \end{equation*}
\end{lemma}

Note that $f$ need not be in the image of the free functor.
We will utilize this result to obtain a rather large number (2970 at level 3 and 7776 at level 4)
of linear equations constraining the GPA coordinates
of the morphisms not living in the image of $\FF_A$.
Thus the half-braid relation will be key to our program, despite not being necessary to prove evaluability.


The source of our algebra objects will be conformal embeddings.
We direct the reader to \cite{DMNO} a more complete treatment.
For a vertex operator algebra $\VV(\gg,j)$, define $\CC(\gg,j) \coloneq \Rep(\VV(\gg,j))$.
% \red{
% \begin{definition}
%     An inclusion $\VV(\gg,j) \subseteq \VV(\hh,k)$ of vertex operator algebras 
%     is said to be {\bf conformal} if the adjoint representation of $\VV(\hh,k)$ restricts to a finite
%     direct sum of simple objects in $\CC(\gg,j) \coloneq \Rep(\VV(\gg,j))$.
% \end{definition}
% }

Affine Lie algebras and conformal embeddings will only be used to obtain algebra objects 
and module fusion graphs, so we briefly recall the correspondence 
\begin{equation}\label{eq:G2-affine-quantum}
    \CC(\gg_2,k) \cong \ol{\Rep(U_{q_k}(\gg_2))} 
\end{equation}
of \cite{cain_autoequivalences}, where $k$ is the level and $q_k$ is given by
\[
    q = e^{\frac{2\pi i}{3(4+k)}}.
\]

From \cite[Appendix]{DMNO} we recall the conformal embeddings which are of use to us:
\begin{equation*}
    \VV(\gg_2,3) \subseteq \VV(\ee_6,1) \quad\text{and}\quad \VV(\gg_2,4) \subseteq \VV(\dd_7,1).
\end{equation*}
At level 3 we have $q_3 = e^{\frac{2\pi i}{42}}$ and at level 4 we have $q_4 = e^{\frac{2\pi i}{48}}$.
We obtain the algebra objects and fundamental graphs for GPAs from \cite{g2_graphs}:
\begin{equation}\label{eq:alg-objetcs}
    A_3 = V_{\emptyset} \oplus V_{\Lambda_1} \quad\text{and}\quad A_4 = V_{\emptyset} \oplus V_{3\Lambda_1}
\end{equation}
at levels 3 and 4, respectively.

Additionally at level 3 and 4, respectively, we have the existence of 
$\Z_3$-like and $\Z_2$-like simple objects $g_3$ and $g_4$ \cite{DMNO}.
% grouplike simple objects 
% \[    
%     g_3 = V_{2\Lambda_1}, \quad\text{and}\quad g_4 = ...
% \]
From \cite{g2_graphs}
we see that at both levels $k=3,4$ we have
\[
    \dim\Hom_{ \ol{\Rep(U_{q_k}(\gg_2))}_{A_k} }(\FF_{A_k}(V_{\Lambda_1})^{\otimes2} \to g_k) = 1.
\]

\begin{remark}\label{rem:Pg-properties}
    It follows that there are idempotents 
    \[
        P_{g_k} : \FF_{A_k}(V_{\Lambda_1})^{\otimes2} \to \FF_{A_k}(V_{\Lambda_1})^{\otimes2}
    \]       
    projecting onto these grouplike objects.
    As the $g_i$ are simple, we have $P_{g_k}^\dagger = P_{g_k}$.

    The behavior of the $P_{g_k}$ will be captured by $\skein{/skein_figs/dec_tet_5}{0.1}$ in $\DD_k$.
    As we will see, describing the interaction of $P_{g_k}$ with the image of the free functor 
    will be sufficient to fully describe $\ol{\Rep(U_{q_k}(\gg_2))}_{A_k}$.
\end{remark}





\subsection{Unoriented Planar Algebras}
Recall the theory of {\bf rigid} monoidal categories detailed in \cite{KW}. 
To put it succinctly, rigid monoidal categories have duals. 
Duals, and the associated evaluation and coevaluation maps, give us cups and caps. 
We also assume pivotality throughout, which gives us the ability to isotope diagrams. 
The generators we will use for our planar algebras will be symmetrically self-dual.

Let $X$ be a (symmetrically self-dual) {\bf tensor generator} for the tensor category $\CC$; 
that is, every object of $\CC$ is isomorphic to a subobject of some tensor power $X^{\otimes n}$. 
Let $\PP_{X;\CC}$ be the full subcategory of $\CC$ whose objects are tensor powers $\unit=X^{\otimes0},X,X^{\otimes2},\dots$; 
we call this the (unoriented) {\bf planar algebra} generated by $X$ in $\CC$. 
The planar algebra $\PP_{X;\CC}$ is {\bf evaluable} if $\dim\End_{\PP_{X;\CC}}(\unit)=1$. 

We will be presenting the our two quantum subgroups as extensions of $\GG_2(q)$ skein theories, 
in the spirit of Kuperberg \cite{Kuperberg,tricats}. 
Up to a rescaling by a factor of $\kappa = \sqrt{[7]-1}$ we use the same skein theory as \cite{tricats}
(note the sign error in the \ref{eq:Pentagon} relation of \cite{Kuperberg}). 
\begin{definition}
    For $q$ a root of unity, the $\GG_2(q)$ skein theory is defined to be that generated by an 
    unoriented trivalent vertex $\skein{/skein_figs/trivalent}{0.05}$ satisfying the relations
    \begin{equation*}\tag{Loop}
        \skein{/skein_figs/loop}{0.1} = \delta = q^{10} + q^{8} + q^{2} + 1 + q^{-2} + q^{-8} + q^{-10}
    \end{equation*}

    \begin{equation*}\tag{Lollipop}\label{eq:Lolli}
        \skein{/skein_figs/lollipop}{0.1} = 0  
    \end{equation*}

    \begin{equation*}\tag{Rotate}\label{eq:Rotate}
        r^1\left( \skein{/skein_figs/trivalent}{0.1} \right) = \skein{/skein_figs/trivalent}{0.1}
    \end{equation*}

    \begin{equation*}\tag{Bigon}\label{eq:Bigon}
        \skein{/skein_figs/bigon}{0.1} = \kappa^2 \skein{/skein_figs/stick}{0.1}
    \end{equation*}

    \begin{equation*}\tag{Trigon}\label{eq:Trigon}
        \skein{/skein_figs/trigon_LHS}{0.1} = -(q^4 +1+ q^{-4}) \skein{/skein_figs/trivalent}{0.1}
    \end{equation*}

    \begin{equation*}\tag{Tetragon}\label{eq:Tetragon}
        \skein{/skein_figs/tet_LHS}{0.1} 
        = (q^2 + q^{-2}) \left( \skein{/skein_figs/dec_tet_1}{0.1} 
        + \skein{/skein_figs/dec_tet_2}{0.1} \right) 
        + (q^2 +1+ q^{-2}) \left( \skein{/skein_figs/dec_tet_3}{0.1} 
        + \skein{/skein_figs/dec_tet_4}{0.1} \right)
    \end{equation*}

    \begin{equation*}\tag{Pentagon}\label{eq:Pentagon}
        \skein{/skein_figs/pentagon}{0.1} = 
        - \sum_{i=0}^4 r^i \left( \skein{/skein_figs/dec_pent_RHS1}{0.1} \right) 
        - \sum_{i=0}^4 r^i \left( \skein{/skein_figs/dec_pent_RHS2}{0.1} \right)
    \end{equation*}
\end{definition}


Our use of planar algebras will depend entirely on the construction of the Cauchy completion,
which we sketch here.
See \cite{cain_noah} for more details and \cite{tuba_wenzl} for a full treatment of the topic.
Recall that the {\bf idempotent completion} of a pivotal tensor category $\CC$ consists of pairs $(Z,p)$,
where $p\in\End_\CC(Z)$ is an idempotent.
We denote the idempotent completion of $\CC$ as $\Idemp(\CC)$.
Further, we define the {\bf additive envelope} of a pivotal, $\C$-linear tensor category $\CC$
to have objects formal direct sums $\bigoplus_j Z_j$ for objects $Z_j$ of $\CC$.
The {\bf Cauchy completion} of $\CC$ is defined by 
\[
    \Ab(\CC) \coloneq \Add(\Idemp(\CC)).
\]

If we again assume $X$ tensor generates $\CC$, it follows that $\CC \cong \Ab(\PP_{X;\CC})$ \cite[Theorem3.4]{tuba_wenzl}.
The universal property of $\Ab(\PP_{X;\CC})$ therefore implies 
that studying $\PP_{X;\CC}$ is sufficient to understand $\CC$. 

By \cite[Corollary 2.20]{bodish_triple_clasp_g2} and \cite[Corollary 6.7]{reconstructing_g2}, the category $\GG_2(q)$ is a {\bf presentation} for the category $\cat{q}$ in the sense that
\[
\ol{\Rep(U_q(\gg_2))} \cong \Ab(\ol{\GG_2(q)}).
\]




\begin{comment}
\subsection{Unitarity and inner products}
Recall that a {\bf dagger} category is one endowed with an involution, denoted by a superscript $\dagger$.
The dagger operation gives us the useful ability to reverse any morphism.
This will be of use when, for example, we need a way to pair parallel {\it or} antiparallel morphisms.

The categorical trace of a morphism $f:X\to X$  is defined by
\[
\tr(f)\coloneqq ev_{X} \circ (f\otimes \id_{X^*}) \circ coev_{X}.
\]
Informally, the trace ``caps off'' an endomorphism and returns the resulting scalar (since we assume $\dim\Hom(\unit\to\unit)\cong\C$).

We use trace to define two different notions of pairing of morphisms:
\[
\ldag f, g\rdag \coloneqq \tr(g^\dagger \circ f)
\]
for parallel morphisms and
\[
\langle \phi,\psi \rangle \coloneqq \tr(\psi\circ\phi)
\]
for antiparallel morphisms. 

Unitarity gives two tools. 
Computationally, it allows us to do computations by ``capping'' diagrams.
It also has useful theoretical consequences.
\begin{definition}
A category $\CC$ is {\bf unitary} if it is a dagger category and the two conditions
\begin{itemize}
    \item if $\ldag f,f\rdag=0$, then $f=0$
    \item for every $f:X\to Y$, there is some $g:X\to X$ such that $ff^\dagger=gg^\dagger$
\end{itemize}
are satisfied.
\end{definition}
On a related note, we also make the following definition
\begin{definition}
A morphism $f:X\to Y$ in a dagger category $\CC$ is {\bf negligible} if $\langle f,g\rangle=0$ for all $g:Y\to X$.
\end{definition}
It is a fact that the negigible morphisms in a category $\CC$ form an ideal, called the {\bf negligible idea} and denoted by $\Neg(\CC)$. We may form the semisimple quotient $\ol{\CC}\coloneqq \CC/\Neg(\CC)$.

\begin{proposition}\label{prop:negligible-induced-functor}
Let $\CC$ and $\DD$ be pivotal, with $\DD$ unitary, and $F:\CC\to\DD$ a pivotal dagger functor. 
Then $F$ descends to a faithful pivotal functor $\ol{F}:\ol{\CC}\to\DD$ such that the diagram
\[
\xymatrix@C=40pt{
\CC \ar[r]^-{F} \ar[d] & \DD \\
\ol{\CC}\ar[ur]^-{\overline{F}} &  \\
}
\]
is commutative.
\end{proposition}

\begin{proof}
    Compute directly:
    \begin{align*}
        \ldag F(f),F(f)\rdag & = \tr(F(f)^\dagger\circ F(f)) \\
        & = \tr(F(f^\dagger)\circ F(f)) \\
        & = F( \tr(f^\dagger \circ f)) \\
        & = F(0)=0
    \end{align*}
    since we assumed $f$ was negligible.
\end{proof}

As one might expect, and as was alluded to above, unitarity interacts nicely with the negligible ideal.

\begin{proposition}
    Let $\CC$ and $\DD$ be unitary, and suppose $F:\CC\to\DD$ is a pivotal dagger functor. 
    Then there is a faithful 
    $\tilde{F}:\ol{\CC} \to \ol{\DD}$
    such that
    \[
        \xymatrix@C=40pt{
        \CC \ar[r]^-{F} \ar[d] & \DD \ar[d] \\
        \ol{\CC}\ar[r]^-{\tilde{F}} & \ol{\DD}  \\
        }
    \]
    If $F$ is faithful, so is $\tilde{F}$.
\end{proposition}

\begin{proof}
    \red{ Compose $\ol{F}$ of Proposition~\ref{prop:negligible-induced-functor} with the functor $\DD \onto \ol{\DD}$. ??}
\end{proof}
\end{comment}







\subsection{Unoriented Graph Planar Algebras}
We will study the quantum subgroups of type $G_2$ by embedding their skein theories into appropriate graph planar algebras (GPAs). 
This serves two purposes:
\begin{itemize}
    \item Giving us solid ground on which to do computations, allowing us to uncover relations by finding them in the GPA hom-spaces, and
    \item Implying unitarity for the skein theories,
\end{itemize}
GPAs are an invention of Vaughan Jones \cite{jones_GPA}.
In this work we have no use for less specialized GPAs, such as the {\it oriented} \cite{Cain_Dan} 
or {\it multi-color} GPA \cite{}, so we consider only the unoriented case. 

\begin{definition}\label{def:GPA}
    Let $\Gamma=(V,E)$ be a finite graph. 
    For an edge $e=(u,v)\in E$, let $\ol{e} \coloneq (v,u)\in E$. 
    The {\bf graph planar algebra} on $\Gamma$, denoted $\GPA(\Gamma)$, is the strictly pivotal rigid monoidal category 
    whose objects are nonnegative integers, and whose hom-spaces have basis
    \[
        \Hom_{\GPA(\Gamma)}(m\to n) \coloneqq \C\left\{ (p,q) \mid \substack{\text{$p$ an $m$-path} \\ \text{$q$ and $n$-path}}, \substack{\text{$s(p)=s(q)$}\\ \text{$t(p)=t(q)$}} \right\},
    \]
    with composition law
    \[
        (p,q)\circ(p',q')\coloneqq \delta_{q=p'} (p,q'),
    \]
    and rigidity maps
    \[
        ev = \sum_e \sqrt{ \frac{\lambda_{t(e)}}{\lambda_{s(e)}} } \langle e\ol{e},s(e) \rangle, 
        \quad coev = \sum_e \sqrt{ \frac{\lambda_{t(e)}}{\lambda_{s(e)}} } \langle s(e) e\ol{e} \rangle
    \]
    where $\lambda$ is the Frobenius-Perron eigenvector of the adjacency matrix of $\Gamma$.
    Monoidal product on objects is addition, and for morphisms is defined by 
    \[
        (p,q)\otimes(p',q')\coloneqq \delta_{s(p')=t(p)} (pp',qq').
    \]
\end{definition}










\begin{comment}
Monoidal functors interact with algebra objects in the ways we would hope.
\begin{proposition}
    Let $\FF:\CC\to\DD$ be a lax-monoidal functor, and let $A$ be an algebra object of $\CC$. 
    Then $\FF A$ has algebra structure induced by $\FF$. 
    If $M$ is a right $A$-module object of $\CC$ then $\FF M$ has a right $\FF A$-moudle structure induced by $\FF$.
\end{proposition}


The following proposition gives us license to make similar statements about adjoints of monoidal functors.
\begin{proposition}
    Monoidal functors between semisimple categories have lax-monoidal right adjoints.
\end{proposition}
\end{comment}

\begin{comment}
In the sequel we will be interested in using facts about the restriction of $\FF_A$ to a planar algebra 
whose Karoubi completion is $\CC$. 
The following fact, which is a restatement of \cite{exactSequencesTensorCategories}[Proposition 5.1], will turn out to be key.

\begin{proposition}\label{prop:exact-functor}
    Let $\FF:\CC\to\DD$ be a monoidal functor with faithful exact right adjoint $R$. If we define $A\coloneqq R(\unit)$, then there is an isomorphism $K$ such that the diagram
    \[
    \xymatrix@C=60pt@R=45pt{
    \CC \ar[r]^{\FF} \ar[dr]^{\FF_A} & \DD \ar[d]^{K} \\
     & \CC_A
     }
    \]
    commutes up to natural isomorphism.
\end{proposition}
\end{comment}










\begin{comment}
\begin{definition}
    An {\bf algebra} object of a braided tensor category $\CC$ is an object $A$ along with two maps
    \begin{align*}
        m: & A\otimes A\to A  \\
        e: & \unit \to A  \\
    \end{align*} 
    such that the following three diagrams 
     \[
    \xymatrix@C=55pt{
    (A\otimes A)\otimes A \ar[d]^{m\otimes \id_A} \ar[r]^{a} & A\otimes(A\otimes A) \ar[r]^{\id_A \otimes m} & A\otimes A \ar[d]^{m} \\
    A\otimes A \ar[rr]^{m}  & & A\\
    }
    \] 
    and 
    \[
    \xymatrix{
    & A \ar[dl] \ar[dr] & \\
    \unit \otimes A \ar[r]^{e} \ar[dr] & A\otimes A \ar[d]^{m} & A\otimes\unit \ar[l]^{e} \ar[dl] \\
    & A & \\
    }
    \] 
    commute (where $a$ is the associator for $\CC$). Given that $\CC$ has braiding $c$, the algebra $A$ is {\bf commutative} if $m\circ c_{A,A}=m$.
    
\end{definition}


Once we have algebra objects one naturally defines modules over them.

\begin{definition}
    Given an algebra object $A$ of $\CC$, an object $M$ of $\CC$ is said to be a {\bf right $A$-module} if there is a map $s:M\otimes A\to A$ such that the diagram
    \[
    \xymatrix{
    (M\otimes A)\otimes A \ar[r]^{a} \ar[d]^{a} & M\otimes(A\otimes A) \ar[r]^{s} & M\otimes A \ar[d]^{s} \\
    M \otimes A \ar[rr]^{s} & & M \\
    }
    \]
    commutes. We define $\CC_A$ to be the category of right $A$-module objects in $\CC$.
\end{definition}
The braiding on $\CC$ allows one to give a right $A$-module the structure of a left $A$-module. Once this has been done, we may define a tensor product on $\CC_A$ by defining $M_1 \otimes_{\CC_A} M_2$ to be the pbject projected onto by the idempotent 
\[
    \skein{/skein_figs/mod_otimes}{0.19}
\]
\end{comment}