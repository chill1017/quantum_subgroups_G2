\section{Preliminaries}
Here we define the players in this game. We begin with planar algebras, in the spirit of \cite{tricats}.



\subsection{Unoriented Planar Algebras}

We begin by recalling the theory of {\bf rigid} monoidal categories detailed in \cite{KW}. 
To put it succintly, rigid monoidal categories have duals. 
Duals, and the associated evaluation and coevaluation maps, give us the cups and caps ubiquitous in planar algebras. 
A rigidity assumption gives us the ability to isotope diagrams. 

Let $X$ be a tensor generator for the tensor category $\CC$; 
that is, every object of $\CC$ is isomorphic to a subobject of some tensor power $X^{\otimes n}$. 
For our pruposes it will suffice to assume $X$ is symmetrically self-dual. 
Let $\PP_{X;\CC}$ be the full subcategory of $\CC$ whose objects are tensor powers $\unit=X^{\otimes0},X,X^{\otimes2},\dots$; 
we call this the (unoriented) {\bf planar algebra} generated by $X$ in $\CC$. 
Since we assumed $X$ tensor generates $\CC$, it follows that $\CC \cong \Kar(\PP_{X;\CC})$, the Cauchy completion of $\PP_{X;\CC}$. 
The universal property of $\Kar(\PP_{X;\CC})$ therefore implies that studying $\PP_{X;\CC}$ is sufficient to understand $\CC$. 
The planar algebra $\PP_{X;\CC}$ is {\bf evaluable} if $\dim\End_{\PP_{X;\CC}}(\unit)=1$. 

\red{Define trivalent cat as generated by a vertex here. Do it fully diagrammatically.}

We will be presenting the our two new categories as extensions of $\GG_2(q)$ skein theories, in the spirit of Kuperberg \cite{Kuperberg}. 
Up to a rescaling by a factor of $k = \sqrt{[7]-1}$ we use the same skein theory (note the sign error in the \ref{eq:Pentagon} relation of \cite{Kuperberg}). 
That is, we define the planar algebra $\GG_2(q)$ generated by a trivalent vertex, with the skein relations 
    \begin{equation*}\tag{Loop}
        \skein{/skein_figs/loop}{0.1} = q^{10} + q^{8} + q^{2} + 1 + q^{-2} + q^{-8} + q^{-10}
    \end{equation*}

    \begin{equation*}\tag{Lollipop}\label{eq:Lolli}
        \skein{/skein_figs/lollipop}{0.1} = 0  
    \end{equation*}

    \begin{equation*}\tag{Rotate}\label{eq:Rotate}
        r\left( \skein{/skein_figs/trivalent}{0.1} \right) = \skein{/skein_figs/trivalent}{0.1}
    \end{equation*}

    \begin{equation*}\tag{Bigon}\label{eq:Bigon}
        \skein{/skein_figs/bigon}{0.1} = k^2 \skein{/skein_figs/stick}{0.1}
    \end{equation*}

    \begin{equation*}\tag{Trigon}\label{eq:Trigon}
        \skein{/skein_figs/trigon}{0.1} = -(q^4 +1+ q^{-4}) \skein{/skein_figs/trivalent}{0.15}
    \end{equation*}

    \begin{equation*}\tag{Tetragon}\label{eq:Tetragon}
        \skein{/skein_figs/square}{0.1} = (q^2 + q^{-2}) \left( \skein{/skein_figs/eye_map}{0.1} + \skein{/skein_figs/eye_rot}{0.1} \right) + (q^2 +1+ q^{-2}) \left( \skein{/skein_figs/capcup}{0.1} + \skein{/skein_figs/doubleStick}{0.1} \right)
    \end{equation*}

    \begin{equation*}\tag{Pentagon}\label{eq:Pentagon}
        \skein{/skein_figs/pentagon}{0.1} = - \sum_{i=0}^4 r^i \left( \skein{/skein_figs/pent_1}{0.15} \right) - \sum_{i=0}^4 r^i \left( \skein{/skein_figs/pent_2}{0.1} \right)
    \end{equation*}
Here, the notation $r^i(\EE)$ means an $i$-click right rotation of the diagram $\EE$. 
For instance, 
\[
r^1\left( \skein{/skein_figs/doubleStick}{0.07} \right) = \skein{/skein_figs/capcup}{0.07} 
\quad\text{and}\quad 
r^2\left( \skein{/skein_figs/doubleStick}{0.07} \right) = \skein{/skein_figs/doubleStick}{0.07}.
\] 
The value of $q$ is given in \cite{cain_autoequivalences} and depends on the level $k$:
\[
q = e^{\frac{2\pi i}{3(4+k)}}.
\]
Thus at level 3 we have $q_3 = e^{\frac{2\pi i}{42}}$ and at level 4 we have $q_4 = e^{\frac{2\pi i}{48}}$.




\subsection{Unitarity and inner products}
Recall that a {\bf dagger} category is one endowed with an involution, denoted by a superscript $\dagger$.
The dagger operation gives us the useful ability to reverse any morphism.
This will be of use when, for example, we need a way to pair parallel {\it or} antiparallel morphisms.

The categorical trace of a morphism $f:X\to X$  is defined by
\[
\tr(f)\coloneqq ev_{X} \circ (f\otimes \id_{X^*}) \circ coev_{X}.
\]
Informally, the trace ``caps off'' an endomorphism and returns the resulting scalar (since we assume $\dim\Hom(\unit\to\unit)\cong\C$).

We use trace to define two different notions of pairing of morphisms:
\[
\ldag f, g\rdag \coloneqq \tr(g^\dagger \circ f)
\]
for parallel morphisms and
\[
\langle \phi,\psi \rangle \coloneqq \tr(\psi\circ\phi)
\]
for antiparallel morphisms. 

Unitarity gives two tools. 
Computationally, it allows us to do computations by ``capping'' diagrams.
It also has useful theoretical consequences.
\begin{definition}
A category $\CC$ is {\bf unitary} if it is a dagger category and the two conditions
\begin{itemize}
    \item if $\ldag f,f\rdag=0$, then $f=0$
    \item for every $f:X\to Y$, there is some $g:X\to X$ such that $ff^\dagger=gg^\dagger$
\end{itemize}
are satisfied.
\end{definition}
On a related note, we also make the following definition
\begin{definition}
A morphism $f:X\to Y$ in a dagger category $\CC$ is {\bf negligible} if $\langle f,g\rangle=0$ for all $g:Y\to X$.
\end{definition}
It is a fact that the negigible morphisms in a category $\CC$ form an ideal, called the {\bf negligible idea} and denoted by $\Neg(\CC)$. We may form the semisimple quotient $\ol{\CC}\coloneqq \CC/\Neg(\CC)$.

\begin{proposition}\label{prop:negligible-induced-functor}
Let $\CC$ and $\DD$ be pivotal, with $\DD$ unitary, and $F:\CC\to\DD$ a pivotal dagger functor. 
Then $F$ descends to a faithful pivotal functor $\ol{F}:\ol{\CC}\to\DD$ such that the diagram
\[
\xymatrix@C=40pt{
\CC \ar[r]^-{F} \ar[d] & \DD \\
\ol{\CC}\ar[ur]^-{\overline{F}} &  \\
}
\]
is commutative.
\end{proposition}

\begin{proof}
    Compute directly:
    \begin{align*}
        \ldag F(f),F(f)\rdag & = \tr(F(f)^\dagger\circ F(f)) \\
        & = \tr(F(f^\dagger)\circ F(f)) \\
        & = F( \tr(f^\dagger \circ f)) \\
        & = F(0)=0
    \end{align*}
    since we assumed $f$ was negligible.
\end{proof}

As one might expect, and as was alluded to above, unitarity interacts nicely with the negligible ideal.

\begin{proposition}
    Let $\CC$ and $\DD$ be unitary, and suppose $F:\CC\to\DD$ is a pivotal dagger functor. 
    Then there is a faithful 
    $\tilde{F}:\ol{\CC} \to \ol{\DD}$
    such that
    \[
        \xymatrix@C=40pt{
        \CC \ar[r]^-{F} \ar[d] & \DD \ar[d] \\
        \ol{\CC}\ar[r]^-{\tilde{F}} & \ol{\DD}  \\
        }
    \]
    If $F$ is faithful, so is $\tilde{F}$.
\end{proposition}

\begin{proof}
    \red{ Compose $\ol{F}$ of Proposition~\ref{prop:negligible-induced-functor} with the functor $\DD \onto \ol{\DD}$. ??}
\end{proof}








\subsection{The Unoriented Graph Planar Algebra}
We will study the quantum subgroups of type $G_2$ by embedding their skein theories into appropriate graph planar algebras. This serves two purposes:
\begin{itemize}
    \item Giving us solid ground on which to do computations, allowing us to uncover relations by finding them in the GPA hom-spaces, and
    \item Implying some nice general properties for the quantum subgroups (i.e., unitarity)
\end{itemize}
In this work we have no use for generalized GPAs, such as the {\it oriented} \cite{Cain_Dan} or {\it multi-color} GPA, so we consider only the unoriented case. 

\begin{definition}\label{def:GPA}
    Let $\Gamma=(V,E)$ be a finite graph. For an edge $e=(u,v)\in E$, let $\ol{e}=(v,u)\in E$. The {\bf graph planar algebra} on $\Gamma$, denoted $\GPA(\Gamma)$, is the strictly pivotal rigid monoidal category whose objects are nonnegative integers, and whose hom-spaces have basis
    \[
        \Hom_{\GPA(\Gamma)}(m\to n) \coloneqq \C\left\{ (p,q) \mid \substack{\text{$p$ an $m$-path} \\ \text{$q$ and $n$-path}}, \substack{\text{$s(p)=s(q)$}\\ \text{$t(p)=t(q)$}} \right\},
    \]
    with composition law
    \[
        (p,q)\circ(p',q')\coloneqq \delta_{q=p'} (p,q'),
    \]
    and rigidity maps
    \[
        ev = \sum_e \sqrt{ \frac{\lambda_{t(e)}}{\lambda_{s(e)}} } \langle e\ol{e},s(e) \rangle, \quad coev = \sum_e \sqrt{ \frac{\lambda_{t(e)}}{\lambda_{s(e)}} } \langle s(e) e\ol{e} \rangle. 
    \]
    Monoidal product on objects is addition, and for morphisms is defined by 
    \[
        (p,q)\otimes(p',q')\coloneqq \delta_{s(p')=t(p)} (pp',qq').
    \]
\end{definition}


We will be finding GPA embeddings of certain planar algebras. 
The following implies unitarity of these planar algebras.
\begin{lemma}
    For $\Gamma$ a finite graph, $\GPA(\Gamma)$ is unitary.
\end{lemma}

\begin{proof}
    
\end{proof}





\subsection{Algebra and Module Objects}
Our ultimate goal is to find skein theoretic descriptions of two categories 
$\cat{q_3}_{A_3}$ and $\cat{q_4}_{A_4}$ 
of modules over algebra objects $A_3$ and $A_4$ coming from the conformal embeddings 
$\hat{\gg_2}\subseteq \hat{\ee_6}$ and $\hat{\gg_2}\subseteq \hat{\dd_7}$. 
In this subsection we define algebra and module objects. 
See \cite{ostrik2001modulecategoriesweakhopf} for a full description. 
Unless otherwise stated, we will be studying braided tensor categories.

\begin{definition}
    An {\bf algebra} object of a braided tensor category $\CC$ is an object $A$ along with two maps
    \begin{align*}
        m: & A\otimes A\to A  \\
        e: & \unit \to A  \\
    \end{align*} 
    such that the following three diagrams 
     \[
    \xymatrix@C=55pt{
    (A\otimes A)\otimes A \ar[d]^{m\otimes \id_A} \ar[r]^{a} & A\otimes(A\otimes A) \ar[r]^{\id_A \otimes m} & A\otimes A \ar[d]^{m} \\
    A\otimes A \ar[rr]^{m}  & & A\\
    }
    \] 
    and 
    \[
    \xymatrix{
    & A \ar[dl] \ar[dr] & \\
    \unit \otimes A \ar[r]^{e} \ar[dr] & A\otimes A \ar[d]^{m} & A\otimes\unit \ar[l]^{e} \ar[dl] \\
    & A & \\
    }
    \] 
    commute (where $a$ is the associator for $\CC$). Given that $\CC$ has braiding $c$, the algebra $A$ is {\bf commutative} if $m\circ c_{A,A}=m$.
    
\end{definition}


Once we have algebra objects one naturally defines modules over them.

\begin{definition}
    Given an algebra object $A$ of $\CC$, an object $M$ of $\CC$ is said to be a {\bf right $A$-module} if there is a map $s:M\otimes A\to A$ such that the diagram
    \[
    \xymatrix{
    (M\otimes A)\otimes A \ar[r]^{a} \ar[d]^{a} & M\otimes(A\otimes A) \ar[r]^{s} & M\otimes A \ar[d]^{s} \\
    M \otimes A \ar[rr]^{s} & & M \\
    }
    \]
    commutes. We define $\CC_A$ to be the category of right $A$-module objects in $\CC$.
\end{definition}
The braiding on $\CC$ allows one to give a right $A$-module the structure of a left $A$-module. Once this has been done, we may define a tensor product on $\CC_A$ by defining $M_1 \otimes_{\CC_A} M_2$ to be the pbject projected onto by the idempotent 
\[
    \skein{/skein_figs/mod_otimes}{0.19}
\]


Monoidal functors interact with algebra 
% and module 
objects in the ways we would hope.

\begin{proposition}
    Let $\FF:\CC\to\DD$ be a lax-monoidal fucntor, and let $A$ be an algebra object of $\CC$. Then $\FF A$ has algebra structure induced by $\FF$. 
    % If $M$ is a right $A$-module object of $\CC$ then $\FF M$ has a right $\FF A$-moudle structure induced by $\FF$.
\end{proposition}
\begin{proof}
    The approach is to use the lax-monoidal structure on $\FF$ and its functoriality to translate diagrams in $\CC$ to diagrams in $\DD$. See \cite{monoidalFunctorsAndAlgebras} for the full argument.
\end{proof}
The following proposition gives us license to make similar statements about adjoints of monoidal functors.
\begin{proposition}
    Monoidal functors between semisimple categories have lax-monoidal right adjoints.
\end{proposition}



The free module functor $\FF_A:\CC\to\CC_A$ given by $X\mapsto X\otimes A$ has a monoidal structure induced by the braiding on $\CC$. Its right adjoint is given by the forgetful functor $\FF^\vee:\CC_A\to\CC$ which is the identity on objects. In the sequel we will be interested in using facts about the restriction of $\FF_A$ to a planar algebra whose Karoubi completion is $\CC$. The following fact, which is a restatement of \cite{exactSequencesTensorCategories}[Proposition 5.1], will turn out to be key.

\begin{proposition}\label{prop:exact-functor}
    Let $\FF:\CC\to\DD$ be a monoidal functor with faithful exact right adjoint $R$. If we define $A\coloneqq R(\unit)$, then there is an isomorphism $K$ such that the diagram
    \[
    \xymatrix@C=60pt@R=45pt{
    \CC \ar[r]^{\FF} \ar[dr]^{\FF_A} & \DD \ar[d]^{K} \\
     & \CC_A
     }
    \]
    commutes up to natural isomorphism.
\end{proposition}




\begin{remark}
    Affine Lie algebras and conformal embeddings will only be used to obtain algebra objects for the quantum subgroups, so we briefly recall the correspondence 
    \[
        \CC(\gg,k) \cong \ol{\Rep(U_q(\gg))}
    \]
    of \cite{}, where $k$ is the {\bf level}. For our purposes this fact translates to 
    \begin{equation}\label{eq:G2-affine-quantum}
        \CC(\gg_2,k) \cong \ol{\Rep(U_{e^{\frac{2\pi i}{3(4+k)}}}(\gg_2))} \cong \ol{\Kar(\GG_2(e^{\frac{2\pi i}{3(4+k)}}))}
    \end{equation}
    as given in \cite{cain_autoequivalences}.
    
    We find the algebra objects and fundamental graphs for GPAs from \cite{g2_graphs}.
\end{remark}