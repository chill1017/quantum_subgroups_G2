\section{Introduction}

This paper gives generators-and-relations presentations for two categories 
\[
    \cat{q_3}_{A_3} \quad \text{and} \quad \cat{q_4}_{A_4}
\] 
where $A_3$ and $A_4$ are the commutative Etale algebras corresponding, respectively, 
to the conformal embeddings $G_{2,3} \subseteq E_{6,1}$ and $G_{2,4} \subseteq D_{7,1}$ 
of affine Lie algebras. 
The correspondence $\CC(\gg,k) \xleftrightarrow{} \ol{\Rep(U_{q_k}(\hh))}$ between 
affine Lie algebra representations and quantum group representations \red{need citation} allows one, 
in the course of studying conformal embeddings, to move freely from the perspective of 
affine Lie algebras to that of quantum groups. 
We will initialize our statements in the language of affine Lie algebras, 
and then translate into quantum groups.

By {\it presentation} we mean the following standard sequence of steps. 
Let $\PP$ be the planar algebra generated by some object of the tensor category $\CC$. 
Then we show $\PP$ is monoidally equivalent to some skein category $\EE$, and that $\Ab(\EE)$ is monoidally equivalent to $\CC$. 
The presentations we give are extensions of Kuperberg's $\GG_2(q)$ skein theory by strands of type $\Z_3$ (level 3) and $\Z_2$ (level 4). 

The novel methodology offered here is the use of graph planar algebra (GPA) embeddings to deduce relations. 
Computing, e.g., the relation (Change of Basis) purely in the context of quantum group representations is daunting, to say the least. 
With a GPA embedding in hand, however, this becomes an elementary linear algebra problem. 
This approach is counter to the usual approach take by, say, Copeland and Edie-Michell in \cite{Cain_Dan}.
Those authors move from generators-and-relations presentations to GPA embeddings for the purpose of 
constructing module categories over $\ol{\Rep(U_q(\sl_N))}$ categories.

Even work with the same aim as this paper, e.g., presenting categories $\CC_A$ of algebras has, thus far,
taken a markedly different form.
The closest example is the recent work of Edie-Michell and Snyder \cite{cain_noah_hans}, which gives presentations 
for categories of the form $\ol{\Rep(U_q(\sl_N))}_A$.
The methods used by these authors is purely representation theoretic;
that is, all of the skein relations for those categories can be deduced from first principles.
At this time there seems to be no clear path from pure representation theory to the presentations we give.
More tools are needed.
Enter GPA embeddings.

% Harder than cain and noah's case becasue I needed to find more relns.
% So this uses GPA emb to find more relns

The present work is organized as follows.
Section~\ref{sec:prelim} sets up most of the theory needed, referencing that which we do not exposit here.
This includes unoriented planar algebras, unoriented graph planar algebras, 
internal algebra and module objects, and some assorted theoretical devices and results.
Section~\ref{sec:skein} then goes on to define a general class of categories of which our examples are members.
We expect this class of categories to be of great use for those researchers intent on conjuring 
examples of exotic new tensor categories.
We then show this class of categories to be evaluable.
Section~\ref{sec:methods} then goes on to discuss what the details of the process of using GPA 
embeddings looks like.
Subsection~\ref{subsec:triv-vertex} details the techniques used to arrive at GPA embeddigns of 
trivalent categories. 
Subsection~\ref{subsec:proj-relns} shows how we go beyond these embeddings of trivalent categories,
and how we use GPA embeddings to explore relations in extensions of trivalent categories.
Finally, Sections~\ref{sec:results-3} and \ref{sec:results-4} give the structure constants for the 
newly constructed categories.
Section~\ref{sec:results-3} details the argument used to prove that we truly have found full presentations.
This covariant Galois-esque argument also appears in \cite{cain_noah_hans}.